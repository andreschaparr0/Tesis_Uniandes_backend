\chapter{Prompts utilizados}

Este apéndice documenta los prompts completos utilizados en el sistema para la estructuración de hojas de vida, descripciones de trabajo y evaluación de compatibilidad. Todos los prompts emplean una arquitectura de dos mensajes: un \textbf{System Message} (que define el rol del modelo) y un \textbf{Human Message} (que contiene la instrucción específica y el texto a procesar).


\section*{Estructuración de CVs}

\subsection*{System Message (Rol común para todos los prompts de CV)}
\begin{tcolorbox}[colback=gray!5!white, colframe=gray!75!black, title=System Message]
\texttt{Eres un experto en extraer información estructurada de hojas de vida. IMPORTANTE: Responde ÚNICAMENTE con la estructura que se te pide, sin texto adicional, sin explicaciones, sin bloques de código markdown (```json).}
\end{tcolorbox}

\textit{Nota:} Cada prompt se complementa con \texttt{Texto del CV: \{text\}}, donde \texttt{\{text\}} es el texto limpio extraído del PDF.


\subsection*{Prompts específicos por campo}

\begin{table}[H]
  \centering
  \caption{Prompts completos de estructuración de CV}
  \label{tab:prompts-cv}
  \scriptsize
  \begin{tabular}{@{}lp{10cm}@{}}
    \toprule
    \textbf{Campo} & \textbf{Prompt (Human Message)} \\
    \midrule
    
    \textbf{Personal} & 
    Extrae únicamente la información personal básica del siguiente CV. Responde en formato JSON con esta estructura exacta: \texttt{\{``name'': ``nombre completo'', ``email'': ``correo electrónico'', ``phone'': ``número de teléfono'', ``location'': ``ubicación/ciudad''\}}. Si no encuentras alguna información, deja el campo como DESCONOCIDO (\texttt{``DESCONOCIDO''}). \\
    \addlinespace
    
    \textbf{Educación} & 
    Extrae únicamente la información educativa del siguiente CV. Responde en formato JSON como un array de objetos con esta estructura: \texttt{[ \{``degree'': ``título obtenido'', ``institution'': ``nombre de la institución'', ``year'': ``año de graduación'', ``field'': ``campo de estudio''\} ]}. Si no hay información educativa, devuelve un array vacío \texttt{[]}. \\
    \addlinespace
    
    \textbf{Experiencia} & 
    Extrae únicamente la experiencia laboral del siguiente CV. Responde en formato JSON como un array de objetos con esta estructura: \texttt{[ \{``position'': ``cargo o puesto'', ``company'': ``nombre de la empresa'', ``duration'': ``duración del trabajo'', ``description'': ``descripción breve de responsabilidades''\} ]}. Si no hay experiencia laboral, devuelve un array vacío \texttt{[]}. \\
    \addlinespace
    
    \textbf{Habilidades técnicas} & 
    Extrae únicamente las habilidades técnicas (technical\_skills) del siguiente CV. Responde en formato array \texttt{[]} de strings: \texttt{[``habilidad1'', ``habilidad2'', ``habilidad3'']}. Si no hay habilidades técnicas, devuelve un array vacío \texttt{[]}. \\
    \addlinespace
    
    \textbf{Habilidades blandas} & 
    Extrae únicamente las habilidades blandas (soft\_skills) del siguiente CV. Responde en formato array \texttt{[]} de strings: \texttt{[``habilidad1'', ``habilidad2'', ``habilidad3'']}. Si no hay habilidades blandas, devuelve un array vacío \texttt{[]}. \\
    \addlinespace
    
    \textbf{Certificaciones} & 
    Extrae únicamente las certificaciones del siguiente CV. Responde en formato JSON como un array de objetos con esta estructura: \texttt{[ \{``name'': ``nombre de la certificación'', ``issuer'': ``institución que la emitió'', ``year'': ``año de obtención''\} ]}. Si no hay certificaciones, devuelve un array vacío \texttt{[]}. \\
    \addlinespace
    
    \textbf{Idiomas} & 
    Extrae únicamente los idiomas del siguiente texto. Responde en formato JSON como un objeto donde las llaves son los idiomas y los valores son los niveles: \texttt{\{``idioma1'': ``nivel1'', ``idioma2'': ``nivel2''\}}. Si no hay idiomas, devuelve un objeto vacío \texttt{\{\}}. \\
    
    \bottomrule
  \end{tabular}
\end{table}



\section*{Estructuración de Jobs}

\subsection*{System Message (Rol común para todos los prompts de Job)}
\begin{tcolorbox}[colback=gray!5!white, colframe=gray!75!black, title=System Message]
\texttt{Eres un experto en extraer información estructurada de descripciones de trabajo. IMPORTANTE: Responde ÚNICAMENTE con la estructura que se te pide, sin texto adicional, sin explicaciones, sin bloques de código markdown (```json).}
\end{tcolorbox}
\textit{Nota:} Cada prompt se complementa con \texttt{Texto del CV: \{text\}}, donde \texttt{\{text\}} es el texto limpio extraído de la descripción de trabajo.
\subsection*{Prompts específicos por campo}

\begin{table}[H]
  \centering
  \caption{Prompts completos de estructuración de descripciones de trabajo}
  \label{tab:prompts-job}
  \scriptsize
  \begin{tabular}{@{}lp{10cm}@{}}
    \toprule
    \textbf{Campo} & \textbf{Prompt (Human Message)} \\
    \midrule
    
    \textbf{Información básica} & 
    Extrae únicamente la información básica del trabajo del siguiente texto. Responde en formato JSON con esta estructura exacta: \texttt{\{``job\_title'': ``título del puesto'', ``company\_name'': ``nombre de la empresa'', ``work\_modality'': ``modalidad de trabajo (presencial, remoto, híbrido)'', ``contract\_type'': ``tipo de contrato (tiempo completo, medio tiempo, etc.)'', ``salary'': ``información salarial'', ``summary'': ``resumen breve del puesto''\}}. Si no encuentras alguna información, deja el campo como DESCONOCIDO (\texttt{``DESCONOCIDO''}). \\
    \addlinespace
    
    \textbf{Responsabilidades} & 
    Extrae únicamente las responsabilidades y funciones del cargo del siguiente texto. Responde en formato array \texttt{[]} de strings: \texttt{[``responsabilidad1'', ``responsabilidad2'', ``responsabilidad3'']}. Si no hay responsabilidades, devuelve un array vacío \texttt{[]}. \\
    \addlinespace
    
    \textbf{Ubicación} & 
    Extrae únicamente la ubicación del trabajo del siguiente texto. Responde en formato JSON con esta estructura: \texttt{\{``location'': ``ubicación del trabajo''\}}. Si no encuentras la ubicación, devuelve \texttt{\{``location'': ``DESCONOCIDO''\}}. \\
    \addlinespace
    
    \textbf{Educación} & 
    Extrae únicamente los requisitos educativos del siguiente texto. Responde en formato JSON con esta estructura: \texttt{\{``education'': ``requisitos educativos''\}}. Si no encuentras requisitos educativos, devuelve \texttt{\{``education'': ``DESCONOCIDO''\}}. \\
    \addlinespace
    
    \textbf{Experiencia} & 
    Extrae únicamente los requisitos de experiencia del siguiente texto. Responde en formato JSON con esta estructura: \texttt{\{``experience'': ``requisitos de experiencia''\}}. Si no encuentras requisitos de experiencia, devuelve \texttt{\{``experience'': ``DESCONOCIDO''\}}. \\
    \addlinespace
    
    \textbf{Habilidades técnicas} & 
    Extrae únicamente las habilidades técnicas requeridas del siguiente texto. Responde en formato array \texttt{[]} de strings: \texttt{[``habilidad1'', ``habilidad2'', ``habilidad3'']}. Si no hay habilidades técnicas, devuelve un array vacío \texttt{[]}. \\
    \addlinespace
    
    \textbf{Habilidades blandas} & 
    Extrae únicamente las habilidades blandas requeridas del siguiente texto. Responde en formato array \texttt{[]} de strings: \texttt{[``habilidad1'', ``habilidad2'', ``habilidad3'']}. Si no hay habilidades blandas, devuelve un array vacío \texttt{[]}. \\
    \addlinespace
    
    \textbf{Certificaciones} & 
    Extrae únicamente las certificaciones requeridas del siguiente texto. Responde en formato array \texttt{[]} de strings: \texttt{[``certificado1'', ``certificado2'', ``certificado3'']}. Si no encuentras certificaciones, devuelve un array vacío \texttt{[]}. \\
    \addlinespace
    
    \textbf{Idiomas} & 
    Extrae únicamente los requisitos de idiomas del siguiente texto. Responde en formato JSON como un objeto donde las llaves son los idiomas y los valores son los niveles: \texttt{\{``idioma1'': ``nivel1'', ``idioma2'': ``nivel2''\}}. Si no hay requisitos de idiomas, devuelve un objeto vacío \texttt{\{\}}. \\
    \addlinespace
    
    \textbf{Beneficios} & 
    Extrae únicamente los beneficios ofrecidos del siguiente texto. Responde en formato array \texttt{[]} de strings: \texttt{[``beneficio1'', ``beneficio2'', ``beneficio3'']}. Si no hay beneficios, devuelve un array vacío \texttt{[]}. \\
    
    \bottomrule
  \end{tabular}
\end{table}



\subsection*{Prompts de inferencia para descripciones }

Estos prompts se activan automáticamente solo cuando la extracción directa no encuentra información.

\begin{table}[H]
  \centering
  \caption{Prompts de fallback para inferencia}
  \label{tab:prompts-job-fallback}
  \scriptsize
  \begin{tabular}{@{}lp{10cm}@{}}
    \toprule
    \textbf{Campo} & \textbf{Prompt (Human Message)} \\
    \midrule
    
    \textbf{Inferencia soft skills} & 
    \textbf{Contexto:} Se activa solo si la extracción directa de soft skills devuelve una lista vacía. \par
    \textbf{Prompt:} Eres un experto en recursos humanos. Analiza las responsabilidades de este trabajo e infiere SOLO las soft skills MÁS CRÍTICAS y NECESARIAS. \par
    Título del trabajo: \texttt{\{job\_title\}} \par
    Responsabilidades: \texttt{- \{responsabilidad1\}, - \{responsabilidad2\}, ...} \par
    IMPORTANTE: Infiere MÁXIMO 3-5 soft skills (no más), si es necesario no pongas ninguna. Solo las MÁS IMPORTANTES y EVIDENTES. NO inventes soft skills genéricas. Basate únicamente en lo que las responsabilidades requieren claramente. \par
    Responde ÚNICAMENTE con un array JSON de strings (nombres de soft skills en español): \texttt{[``soft skill 1'', ``soft skill 2'', ``soft skill 3'']}. Sin bloques de código markdown. \par
     \\
    \addlinespace
    
    \textbf{Detección de idioma} & 
    \textbf{Contexto:} Se activa solo si la extracción directa de idiomas devuelve un objeto vacío. \par
    \textbf{Prompt:} Detecta el idioma principal en el que está escrito el siguiente texto. Responde ÚNICAMENTE con un objeto JSON donde la clave es el nombre del idioma en español y el valor es ``Intermedio'' (como nivel básico requerido). \par
    Formato: \texttt{\{``idioma'': ``Intermedio''\}} \par
    Texto: \texttt{\{primeros 1000 caracteres del texto\}} \par
    Responde solo con el JSON, sin bloques de código markdown. \par
    \\
    
    \bottomrule
  \end{tabular}
\end{table}


\section*{Comparadores}

Los comparadores utilizan prompts de Role Prompting para evaluar la compatibilidad entre CVs y ofertas laborales en distintas dimensiones. A diferencia de los extractores, cada comparador define su propio System Message específico según el aspecto a evaluar.

\textit{Nota:} Todos los comparadores reciben como entrada el CV completo estructurado y la descripción de trabajo completa estructurada en formato JSON.

\subsection*{Prompts de comparadores por aspecto}

\begin{table}[H]
  \centering
  \caption{Prompts completos de comparadores por aspecto}
  \label{tab:prompts-comp}
  \scriptsize
  \begin{tabular}{@{}lp{10cm}@{}}
    \toprule
    \textbf{Aspecto} & \textbf{System Message y Prompt} \\
    \midrule
    
    \textbf{Habilidades técnicas} & 
    \textbf{System Message:} Eres un experto en evaluar compatibilidad de habilidades técnicas entre candidatos y ofertas de trabajo. \par
    \textbf{Prompt:} Compara las habilidades técnicas del CV con las requeridas en la oferta. Considera coincidencias exactas, tecnologías similares y habilidades transferibles. Responde solo con JSON: \texttt{\{``score'': 0--1, ``reason'': ``justificación''\}}. \\
    \addlinespace
    
    \textbf{Experiencia} & 
    \textbf{System Message:} Eres un experto en evaluar experiencia laboral relevante. \par
    \textbf{Prompt:} Evalúa la compatibilidad entre la experiencia del candidato y los requisitos del puesto, considerando años de experiencia, roles previos similares y tecnologías utilizadas. Responde solo con JSON: \texttt{\{``score'': 0--1, ``reason'': ``justificación''\}}. \\
    \addlinespace
    
    \textbf{Responsabilidades} & 
    \textbf{System Message:} Eres un experto en analizar compatibilidad entre responsabilidades de cargo y experiencia profesional. \par
    \textbf{Prompt:} Compara las responsabilidades del puesto con la experiencia del candidato. Evalúa si las funciones previas del candidato se alinean con las responsabilidades requeridas. Responde solo con JSON: \texttt{\{``score'': 0--1, ``reason'': ``justificación''\}}. \\
    \addlinespace
    
    \textbf{Educación} & 
    \textbf{System Message:} Eres un experto en evaluar compatibilidad educativa entre perfiles y requisitos. \par
    \textbf{Prompt:} Evalúa si la formación académica del candidato cumple con los requisitos educativos del puesto. Considera nivel de titulación, campo de estudio y relevancia. Responde solo con JSON: \texttt{\{``score'': 0--1, ``reason'': ``justificación''\}}. \\
    \addlinespace
    
    \textbf{Idiomas} & 
    \textbf{System Message:} Eres un experto en evaluar niveles de idioma y compatibilidad lingüística. \par
    \textbf{Prompt:} Compara los idiomas y niveles de dominio del candidato con los requeridos por el puesto. Evalúa si el nivel del candidato cumple o supera el nivel requerido para cada idioma. Responde solo con JSON: \texttt{\{``compatible'': true/false, ``score'': 0--1, ``reason'': ``justificación''\}}. \\
    \addlinespace
    
    \textbf{Ubicación} & 
    \textbf{System Message:} Eres un experto en evaluar compatibilidad geográfica. \par
    \textbf{Prompt:} Evalúa la proximidad geográfica entre la ubicación del candidato y la ubicación del trabajo. Considera: misma ciudad = 1.0, mismo país = 0.7, países cercanos = 0.3--0.5, países lejanos = 0.0--0.3. Responde solo con JSON: \texttt{\{``score'': 0--1, ``reason'': ``justificación''\}}. \\
    \addlinespace
    
    \textbf{Habilidades blandas} & 
    \textbf{System Message:} Eres un experto en evaluar habilidades blandas y competencias interpersonales. \par
    \textbf{Prompt:} Compara las habilidades blandas del candidato con las requeridas en el puesto. Evalúa coincidencias directas y habilidades complementarias. Responde solo con JSON: \texttt{\{``score'': 0--1, ``reason'': ``justificación''\}}. \\
    \addlinespace
    
    \textbf{Certificaciones} & 
    \textbf{System Message:} Eres un experto en evaluar certificaciones técnicas y profesionales. \par
    \textbf{Prompt:} Evalúa la relevancia y alineación de las certificaciones del candidato con los requisitos o habilidades técnicas del puesto. Considera certificaciones exactas, relacionadas y su vigencia. Responde solo con JSON: \texttt{\{``score'': 0--1, ``reason'': ``justificación''\}}. \\
    
    \bottomrule
  \end{tabular}
\end{table}

% =====================================================
% Tablas de resultados de evaluación del modelo (Fase 1)
% =====================================================

\section*{Resultados de evaluación del modelo - Fase 1}

Las siguientes tablas resumen los resultados cuantitativos de la primera simulación del modelo de comparación por aspectos (vacante Pragma Java Junior), contrastando los \textit{scores} producidos por el modelo y por el evaluador de RR.\,HH. para cada candidato y criterio.

% Tabla 1: Andres Felipe Chaparro Diaz y Juan Esteban Lopez
\begin{table}[H]
    \centering
    \small
    \caption{Evaluación de candidatos - Fase 1 (Parte 1)}
    \label{tab:fase1-eval-parte1}
    \begin{tabular}{|l|c|c|c|c|}
        \hline
        \textbf{Criterio} & \multicolumn{2}{c|}{\textbf{Andrés F. Chaparro}} & \multicolumn{2}{c|}{\textbf{Juan E. López}} \\
        \cline{2-5}
         & \textbf{Score Modelo} & \textbf{Score RR.\,HH.} & \textbf{Score Modelo} & \textbf{Score RR.\,HH.} \\
        \hline
        Experience        & 0\%   & 0\%   & 0\%   & 0\%   \\ \hline
        Technical Skills  & 70\%  & 70\%  & 50\%  & 40\%  \\ \hline
        Education         & 90\%  & 40\%  & 90\%  & 40\%  \\ \hline
        Responsibilities  & 0\%   & 5\%   & 10\%  & 5\%   \\ \hline
        Certifications    & -1    & -1    & -1    & 80\%  \\ \hline
        Soft Skills       & -1    & 86\%  & -1    & 67\%  \\ \hline
        Languages         & -1    & 100\% & -1    & 100\% \\ \hline
        Location          & 0\%   & 100\% & 0\%   & 100\% \\ \hline
        \textbf{Score Final} & \textbf{31\%} & \textbf{35\%} & \textbf{29\%} & \textbf{33\%} \\ \hline
    \end{tabular}
\end{table}

% Tabla 2: Javier Barrera y Gabriel Gomez
\begin{table}[H]
    \centering
    \small
    \caption{Evaluación de candidatos - Fase 1 (Parte 2)}
    \label{tab:fase1-eval-parte2}
    \begin{tabular}{|l|c|c|c|c|}
        \hline
        \textbf{Criterio} & \multicolumn{2}{c|}{\textbf{Javier Barrera}} & \multicolumn{2}{c|}{\textbf{Gabriel Gómez}} \\
        \cline{2-5}
         & \textbf{Score Modelo} & \textbf{Score RR.\,HH.} & \textbf{Score Modelo} & \textbf{Score RR.\,HH.} \\
        \hline
        Experience        & 0\%   & 0\%   & 60\%  & 80\%  \\ \hline
        Technical Skills  & 20\%  & 0\%   & 60\%  & 70\%  \\ \hline
        Education         & 90\%  & 10\%  & 100\% & 100\% \\ \hline
        Responsibilities  & 20\%  & 5\%   & 60\%  & 70\%  \\ \hline
        Certifications    & -1    & -1    & -1    & -1    \\ \hline
        Soft Skills       & -1    & 15\%  & -1    & 67\%  \\ \hline
        Languages         & -1    & 100\% & -1    & 100\% \\ \hline
        Location          & 0\%   & 100\% & 0\%   & 100\% \\ \hline
        \textbf{Score Final} & \textbf{25\%} & \textbf{12\%} & \textbf{65\%} & \textbf{80\%} \\ \hline
    \end{tabular}
\end{table}

% Tabla 3: Juanchito Bernal
\begin{table}[H]
    \centering
    \small
    \caption{Evaluación de candidatos - Fase 1 (Parte 3)}
    \label{tab:fase1-eval-parte3}
    \begin{tabular}{|l|c|c|}
        \hline
        \textbf{Criterio} & \multicolumn{2}{c|}{\textbf{Juanchito Bernal}} \\
        \cline{2-3}
         & \textbf{Score Modelo} & \textbf{Score RR.\,HH.} \\
        \hline
        Experience        & 30\%  & 30\%  \\ \hline
        Technical Skills  & 50\%  & 70\%  \\ \hline
        Education         & 80\%  & 90\%  \\ \hline
        Responsibilities  & 60\%  & 70\%  \\ \hline
        Certifications    & -1    & -1    \\ \hline
        Soft Skills       & -1    & 67\%  \\ \hline
        Languages         & -1    & 100\% \\ \hline
        Location          & 0\%   & 95\%  \\ \hline
        \textbf{Score Final} & \textbf{48\%} & \textbf{62\%} \\ \hline
    \end{tabular}
\end{table}

% Tabla resumen comparativa
\begin{table}[H]
    \centering
    \small
    \caption{Resumen comparativo de scores finales - Fase 1}
    \label{tab:fase1-eval-resumen}
    \begin{tabular}{|l|c|c|}
        \hline
        \textbf{Candidato} & \textbf{Score Modelo} & \textbf{Score RR.\,HH.} \\
        \hline
        Andrés F. Chaparro      & 31\% & 35\% \\ \hline
        Juan E. López           & 29\% & 33\% \\ \hline
        Javier Barrera          & 25\% & 12\% \\ \hline
        Gabriel Gómez           & 65\% & 80\% \\ \hline
        Juanchito Bernal        & 48\% & 62\% \\ \hline
    \end{tabular}
\end{table}
