\chapter{Introducción}
\label{chap:introduccion}

% Contexto y motivación
La gestión eficiente del talento requiere herramientas que permitan relacionar perfiles de candidatos con descripciones de trabajo de forma objetiva, transparente y reproducible. En la práctica, los equipos de reclutamiento enfrentan la revisión manual de grandes volúmenes de hojas de vida (HVs), con formatos heterogéneos (PDFs, imágenes, estilos no estandarizados) y calidad variable de contenido. Este proceso es intensivo en tiempo y recursos, puede introducir sesgos no intencionales en la preselección y dificulta priorizar candidatos óptimos con oportunidad. Este trabajo se enmarca en el nicho de tecnología: el conjunto de HVs utilizado corresponde a perfiles de ingeniería y las descripciones de cargo evaluadas están orientadas a roles de TI y desarrollo de software.

Esta tesis aborda el diseño, implementación y evaluación de un sistema de recomendación que analiza HVs y descripciones de trabajo empleando técnicas de procesamiento de lenguaje natural (NLP) y un esquema de comparación por aspectos. El sistema está compuesto por un backend que procesa y analiza la información, y un frontend que proporciona una interfaz web intuitiva para la gestión de HVs, ofertas de trabajo y visualización de resultados. El sistema toma como entradas una HV (en formato PDF) y una descripción de puesto, extrae y limpia el texto, estructura la información relevante mediante modelos de lenguaje y calcula un \emph{puntaje de ajuste} por distintos aspectos con pesos configurables para reflejar prioridades de negocio. El objetivo es apoyar una preselección inicial más eficiente, objetiva y escalable.

\section{Planteamiento del problema}
\subsection{Situación actual}
La revisión manual de HVs consume una cantidad significativa de tiempo del equipo de reclutamiento y obstaculiza la identificación rápida de candidatos con alto ajuste al puesto. La heterogeneidad de formatos (incluyendo PDFs no indexables o con bajo nivel de estructuración) y la falta de estándares dificulta la automatización y el análisis a escala. Además, la intervención humana temprana puede introducir variabilidad y sesgos no deseados.

\subsection{Situación deseada}
Se busca un prototipo funcional que:
\begin{itemize}
  \item Procese HVs en PDF, extraiga y limpie el texto de forma robusta.
  \item Estructure la información de HVs y descripciones.
  \item Calcule un \emph{puntaje de ajuste} HV--puesto y genere un ranking de candidatos.
  \item Reduzca el tiempo de revisión manual inicial  y aumente la probabilidad de identificar candidatos adecuados en etapas tempranas.
\end{itemize}

\subsection{Restricciones y enfoque}
El desarrollo se realizará con un conjunto de datos inicial de 10 hojas de vida, donde la mayoría corresponden a estudiantes o egresados de la Universidad de los Andes. De estas 10 HVs, 5 pertenecen a estudiantes activos (4 de pregrado y 1 de postgrado). Este conjunto acotado permite adoptar un enfoque iterativo de mejora continua, privilegiando una arquitectura modular, reproducible y auditable que permita refinar comparadores y pesos sin afectar la estabilidad del sistema. La limitación en el tamaño del dataset refleja el alcance inicial del prototipo y permite una evaluación controlada antes de escalar a volúmenes mayores. Adicionalmente, el estudio está acotado al dominio de tecnología; por lo tanto, la generalización de resultados hacia otros sectores (p. ej., salud, finanzas, manufactura) se discute como trabajo futuro y amenaza a la validez externa.

Como refuerzo a la evaluación técnica y para incorporar criterios de negocio propios del dominio, se contó con la participación de un profesional de recursos humanos, Juan Pablo Chaparro (25 años, alrededor de 2 años de experiencia en RR. HH.), quien colaboró en la revisión de resultados y en la interpretación de métricas del sistema. Esta retroalimentación experta contribuyó a contrastar los puntajes generados por el algoritmo con criterios prácticos de selección y a identificar casos límite relevantes para el flujo de reclutamiento.

\section{Pregunta de investigación}
¿En qué medida un sistema de evaluación de hojas de vida basado en LLMs logra alinearse con la clasificación de expertos humanos y reducir los tiempos de revisión en procesos de selección del sector tecnológico?

\section{Objetivos}
\subsection{Objetivo general}
Diseñar, implementar y evaluar un sistema de recomendación de hojas de vida basado en Grandes Modelos de Lenguaje (LLMs), con el fin de mejorar la eficiencia y efectividad del proceso de selección de talento en una empresa.

\subsection{Objetivos específicos}
\begin{enumerate}
  \item Desarrollar una metodología y herramientas para la extracción y estructuración de información a partir de HVs en formato PDF no indexable o poco estructurado.
  \item Diseñar e implementar comparadores por aspecto y un esquema de ponderación configurable para calcular el porcentaje de recomendación.
  \item Evaluar el rendimiento técnico del algoritmo y el impacto potencial del sistema en métricas operacionales de eficiencia y efectividad.
  \item Proponer lineamientos para la integración del sistema en el flujo de trabajo del equipo de reclutamiento.
\end{enumerate}

% Contribuciones
\section{Contribuciones}
\begin{itemize}
  \item Una arquitectura en capas (API, servicios, repositorios y base de datos) que soporta ingestión, análisis y persistencia de resultados, basada en \texttt{FastAPI}, \texttt{SQLAlchemy} y \texttt{SQLite}.
  \item Una interfaz web (frontend) desarrollada en \texttt{React} con \texttt{Vite} y \texttt{TailwindCSS} que permite gestionar HVs, crear ofertas de trabajo, ejecutar análisis de compatibilidad y visualizar resultados mediante gráficos interactivos.
  \item Un módulo de limpieza y extracción de texto de PDFs, y un módulo de estructuración de HVs y descripciones apoyado en orquestación de prompts utilizando \texttt{GPT-4o-mini} de Azure OpenAI.
  \item Un motor de comparación por aspectos con desglose explicativo de puntajes y esquema de pesos configurable.
  \item Un protocolo experimental y un conjunto de métricas para evaluar desempeño técnico y operacional del sistema.
  \item Un componente de validación experta con participación de un profesional de RR. HH. (Juan Pablo Chaparro), para contrastar resultados y ajustar criterios de interpretación.
\end{itemize}

\section{Consideraciones éticas y normativas}
El manejo de datos personales presentes en HVs exige medidas de seguridad, consentimiento informado y cumplimiento de la normativa de protección de datos aplicable. Asimismo, se abordará la mitigación de sesgos algorítmicos mediante transparencia de criterios, explicación de puntajes por aspecto y validaciones con casos de prueba controlados. Estas consideraciones se retoman en detalle al discutir amenazas a la validez y aspectos éticos del estudio.


