\chapter{Introducción}
\label{chap:introduccion}

% Contexto y motivación
La gestión eficiente del talento es un desafío crítico en las organizaciones modernas. En los procesos de selección tradicionales, los equipos de recursos humanos se enfrentan a la revisión manual de grandes volúmenes de hojas de vida (HVs), presentadas en formatos heterogéneos y con niveles de detalle variables. Este filtro inicial es intensivo en tiempo y, a menudo, se convierte en un cuello de botella que retrasa la identificación de los candidatos más idóneos \cite{factorhuma2021}.

En el contexto específico del sector tecnológico, este problema se agudiza debido a la complejidad técnica de los perfiles y la rápida evolución de las habilidades requeridas. La presente tesis aborda esta problemática mediante el diseño, implementación y evaluación de un sistema de recomendación de talento basado en Inteligencia Artificial. A diferencia de los sistemas tradicionales basados en palabras clave \cite{franco2025}, esta propuesta utiliza Grandes Modelos de Lenguaje (LLMs) para interpretar semánticamente tanto las hojas de vida como las descripciones de trabajo, permitiendo una comparación profunda por aspectos clave (experiencia, habilidades, educación, entre otros).

El sistema desarrollado no pretende reemplazar el juicio humano, sino potenciarlo mediante una herramienta. A través de una arquitectura que combina un backend en FastAPI, un frontend en React y servicios de OpenAI, la solución estructura información no estructurada, calcula puntajes de compatibilidad transparentes y ofrece explicaciones detalladas para cada recomendación, facilitando así una toma de decisiones más ágil y fundamentada.

\section{Planteamiento del problema}

\subsection{Situación actual}
Actualmente, el primer filtro de reclutamiento depende en gran medida de la revisión manual o de sistemas automatizados básicos que filtran candidatos buscando coincidencias exactas de palabras clave. Este enfoque presenta limitaciones graves: descarta perfiles calificados simplemente por no usar los términos exactos de la vacante y falla al procesar documentos con diseños complejos o formatos no estándar \cite{jobswift2024}. Como resultado, los reclutadores deben invertir una cantidad excesiva de tiempo en validar información que podría haber sido mal interpretada, reduciendo el espacio para una evaluación más profunda y humana del talento \cite{resumemate2025}.

\subsection{Situación deseada}
Se busca implementar un prototipo funcional capaz de ingerir documentos en formato PDF, extraer y estructurar su información, y realizar un cruce semántico contra los requisitos de una vacante. El sistema debe entregar no solo un ranking de candidatos, sino una justificación desglosada del porqué de cada puntaje, actuando como un asistente inteligente que pre-digiere la información para el reclutador.

\section{Pregunta de investigación}
¿En que medida un sistema de calificación automático de Hojas de Vida mediado por LLMs en el dominio tecnológico se acerca a al ranking dado por un experto humano dada una descripción de trabajo y si este permite reducción de tiempos de revisión?


\section{Objetivos}

\subsection{Objetivo general}
Diseñar, implementar y evaluar un sistema de recomendación de hojas de vida basado en procesamiento de lenguaje natural y modelos LLM, que permita estructurar información no estructurada y calcular la compatibilidad candidato-vacante para mejorar la eficiencia del primer filtro de selección.

\subsection{Objetivos específicos}
\begin{enumerate}
  \item Desarrollar una metodología y herramientas para la extracción y estructuración de
información a partir de HVs en formato PDF y ofertas de trabajo en texto plano.
  \item Diseñar e implementar comparadores por aspecto y un esquema de ponderación configurable para calcular el porcentaje de recomendación.
  \item Evaluar el rendimiento técnico del algoritmo y el impacto potencial del sistema en
métricas operacionales de eficiencia y efectividad.
  \item Proponer lineamientos para la integración del sistema en el flujo de trabajo del equipo
de reclutamiento.

\end{enumerate}

\section{Alcance y delimitaciones}
El desarrollo y la validación del sistema se acotan al dominio de perfiles tecnológicos (desarrolladores, ingenieros de sistemas, científicos de datos). El conjunto de datos experimental consta de 10 hojas de vida reales anonimizadas y 4 descripciones de trabajo de empresas del sector, seleccionadas para desde practicantes hasta perfiles junior.

La validación cualitativa cuenta con la participación de Juan Pablo Chaparro, profesional de Recursos Humanos, quien actúa como evaluador para medir la precisión del modelo. No se contempla el despliegue en producción masiva ni la integración con portales de empleo externos en esta fase de tesis.

\section{Estructura del documento}
El presente documento se organiza en los siguientes capítulos:

\begin{itemize}
    \item \textbf{Capítulo \ref{chap:marco-teorico}: Marco Teórico.} Presenta los fundamentos conceptuales que abordan la tesis como Modelos de Lenguaje Grandes (LLMs), herramientas tecnológicas base como LangChain y más.
    
    \item \textbf{Capítulo \ref{chap:estado-del-arte}: Estado del Arte.} Revisa la literatura existente sobre sistemas de recomendación en reclutamiento, desde enfoques clásicos hasta tendencias modernas con IA, identificando las brechas que este trabajo busca cubrir.
    
    \item \textbf{Capítulo \ref{chap:metodologia}: Metodología.} Detalla el diseño de la investigación, la conformación del dataset, los esquemas de datos definidos para CVs y Jobs, y el protocolo de evaluación utilizado.
    
    \item \textbf{Capítulo \ref{chap:diseno-implementacion}: Diseño, Arquitectura e Implementación.} Describe la construcción técnica del sistema, explicando la arquitectura por capas del backend, los flujos de los servicios y el diseño de la interfaz de usuario.
    
    \item \textbf{Capítulo \ref{chap:evaluaciones}: Experimentos y Resultados.} Presenta los resultados de la evaluación de estructuración y las evaluaciones de procesos de selección, analizando la convergencia entre el modelo y el experto humano en dos fases de experimentación.
    
    \item \textbf{Capítulo \ref{chap:discusion}: Discusión y Amenazas a la Validez.} Falta por hacer...
    
    \item \textbf{Capítulo \ref{chap:conclusiones}: Conclusiones y Trabajo Futuro.} Sintetiza los logros alcanzados respecto a los objetivos planteados y propone líneas de investigación para la evolución del sistema.
\end{itemize}

\section{Consideraciones éticas}
El uso de sistemas automatizados en selección de personal conlleva responsabilidades éticas significativas. Este proyecto aborda la privacidad mediante el uso de datos autorizados con fines académicos. Asimismo, se prioriza la explicabilidad del algoritmo: el sistema no emite un veredicto de caja negra, sino que expone las razones de cada puntaje, permitiendo que el humano mantenga el control final y pueda auditar posibles sesgos algorítmicos.
