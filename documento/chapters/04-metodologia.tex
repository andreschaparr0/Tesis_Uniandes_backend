\usepackage{wrapfig}
\usepackage{graphicx}
\usepackage{float}
\usepackage{caption}
\chapter{Metodología}
\label{chap:metodologia}

\section{Definición del problema, usuarios y datos}

\subsection{Caracterización del proceso de contratación y punto de intervención}
En procesos de contratación tradicionales, la etapa inicial suele consistir en un filtro masivo de hojas de vida (HVs) realizado por equipos de recursos humanos (RR.\,HH.). En esta fase, se reciben decenas o cientos de HVs por vacante y se debe identificar, con información limitada y bajo restricciones de tiempo, un subconjunto de candidatos potencialmente adecuados para continuar a entrevistas técnicas o evaluaciones específicas. Este trabajo propone intervenir de manera focalizada esa primera etapa —sin reemplazar la evaluación humana posterior— ofreciendo un sistema de apoyo que estructure la información de HVs y descripciones de trabajo, compare por aspectos relevantes y entregue un puntaje de compatibilidad con desglose explicativo. El objetivo es reducir el esfuerzo de revisión manual inicial y mejorar la priorización de perfiles, manteniendo criterios transparentes y reproducibles.

\subsection{Perfil del usuario objetivo}
El usuario primario de la aplicación es un profesional de RR.\,HH. responsable del primer filtro de candidatos. Este usuario requiere una herramienta que facilite la ingestión y organización de información, permita ejecutar comparaciones consistentes entre HVs y descripciones de cargo y entregue resultados interpretables para justificar decisiones de preselección. En este contexto, la interfaz y el flujo de trabajo se orientan a: (i) cargar HVs en PDF, (ii) registrar o pegar descripciones de trabajo, (iii) ejecutar análisis por vacante y (iv) consultar rankings y desgloses de puntajes por aspecto para sustentar la priorización.

\subsection{Recolección de hojas de vida (HVs)}
Las HVs utilizadas se recolectaron principalmente en el entorno de la Universidad de los Andes, solicitando voluntariamente documentos a estudiantes y egresados del área de tecnología (algunos con experiencia profesional y otros próximos a graduarse). El conjunto final estuvo compuesto por diez HVs. Dado que el sistema no requiere entrenar modelos desde cero —sino orquestar extracción y comparación mediante modelos de lenguaje— no fue necesario generar datos sintéticos; se privilegió un conjunto acotado que permitiera iterar con rapidez, auditar resultados y controlar la variabilidad en pruebas. Este tamaño de muestra es coherente con el carácter de prototipo y con el objetivo de demostrar la utilidad del apoyo al primer filtro.

\subsection{Recolección de descripciones de trabajo}
Considerando que la mayoría de HVs corresponden a perfiles de tecnología, se recopilaron descripciones de trabajo del mismo nicho (TI y desarrollo de software), buscando diversidad en empresas, roles y niveles de seniority. La selección incluyó ofertas de prácticas y posiciones con hasta aproximadamente tres años de experiencia para cubrir escenarios habituales del mercado junior y semisenior. Este espectro de descripciones permite contrastar el desempeño del sistema en contextos variados —tecnologías, responsabilidades, modalidades y ubicaciones— y evaluar la estabilidad de los desgloses por aspecto en situaciones realistas de preselección.

\section{Estructuración de la información de entrada}

\subsection{Estrategia de estructuración basada en prompts}
Con la definición del problema y los datos, se requiere convertir HVs y descripciones de trabajo en representaciones estructuradas que el sistema pueda comparar de forma consistente. Para ello se emplea \emph{prompt engineering} con un modelo de lenguaje (GPT-4o-mini), asignándole un rol claro y especificando formatos de salida estrictos (JSON). Esta estrategia, detallada técnicamente en el capítulo de Implementación, permite extraer campos clave por rubro de los documentos. Cuando la información no está disponible, se registra como \texttt{DESCONOCIDO} , manteniendo trazabilidad de ausencias sin introducir supuestos no verificados.

\subsection{Esquema de estructuración de hojas de vida}
Para HVs se adopta un esquema orientado a capturar la información mínima necesaria para la preselección temprana en el proceso de selección:

\subsection{Esquema de estructuración de hojas de vida}
Para HVs se adopta un esquema orientado a capturar la información mínima necesaria para la preselección temprana en el proceso de selección:

\vspace{0.3cm}
\noindent
\begin{minipage}[t]{0.62\textwidth}
\vspace{0pt}
\begin{itemize}
    \item \textbf{Información personal}: Corresponde a los datos básicos del candidato (nombre, correo, número de contacto y ubicación de referencia).
    \item \textbf{Educación}: lista de objetos con \texttt{grado}, \texttt{institución}, \texttt{año}, \texttt{campo}. Resume los títulos formales obtenidos y sus campos de estudio.
    \item \textbf{Experiencia}: lista de objetos con \texttt{posición}, \texttt{compañia}, \texttt{duración}, \texttt{descripción}. Describe los cargos desempeñados, el contexto laboral y las responsabilidades principales.
    \item \textbf{Habilidades técnicas}: lista de habilidades técnicas. Incluye tecnologías, lenguajes y herramientas declaradas por el candidato.
    \item \textbf{Habilidades blandas}: lista de habilidades blandas. Agrupa competencias transversales reportadas o inferidas del CV.
    \item \textbf{Certificaciones}: lista de objetos con los certificados. Representa credenciales formales que respaldan conocimientos o habilidades específicas.
    \item \textbf{Idiomas}: objeto \{idioma: nivel\}. Asocia cada lengua al nivel de dominio declarado.
\end{itemize}
\end{minipage}%
\hfill
\begin{minipage}[t]{0.31\textwidth}
    \vspace{0pt}
    \centering
    \begin{figure}[H]
        \centering
        \includegraphics[width=0.95\textwidth]{estructura_cuadros_elemenos_HV.png}
        \caption{Estructura de elementos del CV}
        \label{fig:estructura-cv}
    \end{figure}
\end{minipage}

\vspace{1em}

Esta selección equilibra cobertura y simplicidad: concentra atributos que típicamente evalúan RR.\,HH. en el primer filtro y se alinea con los comparadores por aspecto (experiencia, educación, habilidades, certificaciones, idiomas y ubicación). El formato facilita la explicación de resultados y la detección explícita de ausencias (\texttt{DESCONOCIDO}) sin introducir ruido.

\subsection{Esquema de estructuración de descripciones de trabajo}
Para descripciones de trabajo se define un esquema que recoge requisitos y contexto del cargo, habilitando la comparación por aspectos:

\vspace{0.3cm}
\noindent
\begin{minipage}[t]{0.62\textwidth}
\vspace{0pt}
\begin{itemize}
    \item \textbf{Información básica}: Titulo cargo, nombre empresa, modalidad trabajo, tipo contrato, salario y resumen. Proporciona el marco general del rol.
    \item \textbf{Responsabilidades}: lista de responsabilidades. Tareas y funciones esperadas del cargo.
    \item \textbf{Ubicación}: Referencia geográfica relevante para compatibilidad y modalidad.
    \item \textbf{Educación}: Requisitos educativos mínimos o deseables.
    \item \textbf{Experiencia}: Años o tipo de experiencia requerida.
    \item \textbf{Habilidades técnicas}: lista de habilidades técnicas. Tecnologías y herramientas obligatorias o deseables.
    \item \textbf{Habilidades blandas}: lista de habilidades blandas. Competencias transversales esperadas.
    \item \textbf{Certificaciones}: lista de certificaciones. Certificaciones requeridas o valoradas.
    \item \textbf{Idiomas}: objeto \{idioma: nivel\}. Niveles de idioma requeridos.
    \item \textbf{Beneficios}: lista de beneficios. Condiciones y beneficios relevantes.
\end{itemize}
\end{minipage}%
\hfill
\begin{minipage}[t]{0.35\textwidth}
    \vspace{0pt}
    \centering
    \begin{figure}[H]
        \centering
        \includegraphics[width=0.65\textwidth]{estructura_cuadros_elementos_descripcion.png}
        \caption{Estructura de elementos de la descripción}
        \label{fig:estructura-descripcion}
    \end{figure}
\end{minipage}

\vspace{1em}
Este esquema prioriza la comparabilidad con los campos del CV y la trazabilidad de requisitos típicos del mercado tecnológico (responsabilidades, habilidades, experiencia, educación, ubicación e idiomas). La homogeneización en JSON permite ejecutar comparaciones semánticas y producir un desglose por aspecto comprensible para RR.\,HH.; las ausencias se documentan como \texttt{DESCONOCIDO} para evitar inferencias no sustentadas.

 \section{Comparación por aspectos}
 
 \subsection{Alcance de los elementos comparados}
 Una vez estructuradas las HVs y las descripciones de trabajo, se realiza la comparación por aspectos entre ambas representaciones. No todos los campos se emplean para el cálculo: en particular, la información personal del CV y los beneficios como parte de la información básica del trabajo (p.\,ej., título o salario) se conservan para visualización y contexto, pero no inciden en la compatibilida. El foco está en campos con significado directo para el primer filtro: experiencia, habilidades técnicas y blandas, educación, certificaciones, idiomas y ubicación.
 
 \subsection{Correspondencia de campos y casos especiales}
La comparación se realiza preferentemente entre campos homólogos. Por ejemplo: habilidades técnicas del CV frente a habilidades técnicas de la descripción del trabajo; educación frente a educación; e idiomas frente a idiomas. Como se ilustra en la figura correspondiente, también existen casos particulares como el de responsabilidades, presente en la descripción del trabajo pero no explícitamente en la hoja de vida. Para capturar esta correspondencia funcional, las responsabilidades del trabajo se contrastan con la experiencia de la hoja de vida, dado que las responsabilidades esperadas deben reflejarse en las tareas efectivamente realizadas por el candidato. Detalles adicionales sobre estas comparaciones se desarrollan con mayor profundidad en el capítulo de implementación.

\begin{figure}[H]
    \centering
    \includegraphics[width=0.5\linewidth]{correspondencia_campos.png}
    \caption{Correspondencia entre los rubros de las hojas de vida y los de las descripciones de trabajo.}
    \label{fig:placeholder}
\end{figure}
 
 \subsection{Metodología de comparación y esquema de puntajes}
 Cada comparación se implementa mediante prompt engineering con un LLM (GPT-4o-mini), asignando un rol claro y un formato de salida estrictamente controlado (JSON con score y reason). El \texttt{score} toma valores en \([0,1]\): 1.0 indica alta compatibilidad, valores intermedios reflejan correspondencia parcial y 0.0 indica baja o nula coincidencia; la \texttt{reason} ofrece una justificación breve y específica del resultado. En casos sin información suficiente (p.\,ej., el campo requerido es \texttt{DESCONOCIDO} en alguna de las partes), la comparación devuelve \(-1\) para señalizar no evaluado y una razón explícita. Se adoptan indicaciones conservadoras y validaciones de salida para reducir la incidencia de resultados no evaluables; los detalles de prompts y estrategias de respaldo se documentan en el capítulo de Implementación.
 
 \section{Integración de puntajes y análisis de cobertura}
 
Una vez obtenidos los puntajes por aspecto, se combina un \emph{score} final mediante una suma ponderada. Se adoptan pesos por defecto que reflejan prácticas habituales en la preselección de perfiles tecnológicos y el énfasis en experiencia y capacidades técnicas: experiencia 30\%, habilidades técnicas 15\%, educación 15\%, responsabilidades 15\%, certificaciones 10\%, habilidades blandas 8\%, idiomas 4\% y ubicación 3\%. Este reparto prioriza la evidencia práctica (experiencia y responsabilidades) y la idoneidad técnica, sin descuidar educación y certificaciones; idiomas y ubicación se consideran criterios de menor peso en escenarios típicos de tecnología. Los pesos son configurables para adaptarse a distintos tipos de vacante; sin embargo, en la evaluación de esta tesis se emplearon los valores por defecto para mantener control experimental. El cálculo del \emph{score} final consiste en sumar \(\texttt{score\_aspecto} \times \texttt{peso\_aspecto}\) a través de los aspectos considerados.
 
 \subsection{Normalización ante aspectos no evaluados}
 En el desarrollo se observó que algunos aspectos podían resultar \(-1\) (no evaluados) por falta de datos. Para evitar penalizaciones artificiales, se procede a \emph{normalizar} el \emph{score} final dividiendo la suma ponderada entre la suma de pesos efectivamente usados (es decir, ignorando los aspectos no evaluados). Intuitivamente, si solo se pudo evaluar un subconjunto de aspectos, el \emph{score} final refleja el desempeño en ese subconjunto sin “arrastrar” pesos de aspectos ausentes. Esta decisión favorece comparaciones justas cuando la información es incompleta por causas ajenas al candidato o al anuncio.
 
 \subsection{Limitaciones de la normalización y sesgo por escasez de datos}
 La normalización introduce un efecto colateral: candidatos con poca información (pero favorable en los pocos aspectos evaluados) pueden exhibir un \emph{score} final elevado. Por ejemplo, una HV con solo tres aspectos evaluables (menos del 50\% de cobertura) puede obtener un puntaje alto si esos tres aspectos coinciden fuertemente, aunque el resto falte. Este comportamiento no es un error del método, sino una consecuencia de reportar desempeño condicionado a la información disponible; no obstante, requiere una lectura informada para no sobrevalorar perfiles con evidencia limitada.
 
 \subsection{Visualización de cobertura vs.\ score y umbrales de decisión}
 Para mitigar la interpretación sesgada, se propone una visualización de dispersión donde el eje \(x\) representa el porcentaje de criterios efectivamente evaluados (cobertura) y el eje \(y\) el \emph{score} final normalizado. Esta gráfica permite contextualizar el puntaje respecto al respaldo informativo: un \emph{score} alto con baja cobertura sugiere cautela, mientras que puntajes altos con alta cobertura indican evidencia más robusta. Adicionalmente, se pueden definir umbrales operativos de cobertura (p.\,ej., 60\% y 80\%) para clasificar candidatos en bandas de confianza y priorización. De esta forma, la decisión de avanzar un perfil no depende únicamente del \emph{score}, sino también de la cantidad de criterios evaluados, promoviendo una lectura más equilibrada en la etapa de preselección.
 
 \section{Diseño de la evaluación de la aplicación}
 
 \subsection{Objetivo y alcance de la evaluación}
 Con la aplicación operativa, se evalúa la propuesta respecto a los objetivos de preselección: qué componentes se miden, cómo se miden y quién realiza la valoración. La evaluación se centra en dos frentes: (i) la calidad de la \emph{estructuración} de HVs y descripciones (para asegurar entradas confiables al motor) y (ii) el \emph{modelo de comparación por aspectos} (para verificar utilidad práctica y coherencia con el criterio humano). Se busca caracterizar el desempeño del sistema sin alterar su alcance: primer filtro en perfiles tecnológicos, con un conjunto de datos acotado y controlado.
 
 \subsection{Etapas y datos utilizados}
 La evaluación usa el mismo conjunto de datos empleados durante el desarrollo: diez HVs y cuatro descripciones de trabajo del dominio tecnológico. Se divide en dos etapas: (1) evaluación de estructuración (CVs y Jobs) y (2) evaluación del modelo de comparación. Esto permite trazar una línea clara entre la calidad de las entradas (datos estructurados) y el comportamiento del motor de recomendación (puntajes y desgloses por aspecto).
 
 \subsection{Evaluación de la estructuración (CVs y Jobs)}
 En la primera etapa se valora la calidad de la estructuración tanto de HVs como de descripciones. Un evaluador con conocimiento del contexto del proyecto (sin requerir especialidad en RR.\,HH.) revisa rubro por rubro y califica, para cada documento, qué tan bien la aplicación capturó la información objetivo. La evaluación se realiza sobre las diez HVs y las cuatro descripciones; los resultados se registran en una hoja de cálculo para su consolidación. El detalle del instrumento, los rubros evaluados y los resúmenes de resultados se presentan en el capítulo de \emph{Experimentos y Resultados}.
 
 \subsection{Evaluación del modelo de comparación (criterio de RR.\,HH.)}
 En la segunda etapa se valora el comportamiento del modelo de comparación por aspectos simulando escenarios reales con la información cargada. Se ejecutan cuatro análisis; cada análisis considera una descripción de trabajo y cinco HVs (seleccionadas del conjunto disponible). Un profesional de RR.\,HH., Juan Pablo Chaparro (25 años, \(\sim\)2 años de experiencia), califica por análisis: (i) el \emph{score} de compatibilidad propuesto por el sistema y (ii) las \emph{razones} por aspecto que sustentan ese score. Adicionalmente, emite un ranking de candidatos que se contrasta cualitativamente con el ranking del sistema. El diseño detallado, ejemplos de salidas y el resumen de hallazgos se documentan en el capítulo de \emph{Experimentos y Resultados}.
 
 \section{Síntesis de la metodología}
 La metodología propuesta articula, de forma reproducible y auditable, el encuadre del problema y del usuario objetivo, la recolección controlada de datos (diez HVs y cuatro descripciones del dominio tecnológico), la estructuración basada en \emph{prompts} con salidas JSON estandarizadas (marcando \texttt{DESCONOCIDO} ante ausencias), la comparación por aspectos con justificación textual, y la integración de puntajes mediante una suma ponderada con normalización ante criterios no evaluados. Asimismo, introduce una lectura informada del \emph{score} final a través de la visualización cobertura–puntaje y define un protocolo de evaluación en dos etapas (calidad de estructuración y valoración del modelo por RR.\,HH.) que permite contrastar utilidad práctica y coherencia con el criterio humano. Este diseño metodológico sustenta la validez del enfoque en el primer filtro de selección, a la vez que deja trazado un camino claro para la replicación y el análisis crítico de resultados.
