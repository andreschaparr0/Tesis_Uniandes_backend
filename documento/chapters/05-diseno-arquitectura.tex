\chapter{Diseño y Arquitectura}
\label{chap:diseno}

\section{Visión General}
El sistema adopta una arquitectura en capas que separa responsabilidades para facilitar mantenibilidad, pruebas y escalabilidad futura. Las capas son: API (FastAPI), Servicios, Repositorios, Persistencia (SQLite) y el Motor de Recomendación (comparadores por aspecto + cálculo de \emph{score}).

\section{Arquitectura por capas del backend}
\subsection*{API}
La API (archivo \texttt{api/main.py}) define endpoints REST para gestionar CVs, Jobs y Análisis, valida entradas y compone respuestas. Expone también endpoints de estado y estadísticas.

\subsection*{Servicios}
Los servicios (archivo \texttt{api/services.py}) orquestan el pipeline:
\begin{itemize}
  \item \textbf{CVService}: extrae/limpia texto de PDFs y estructura CVs a JSON.
  \item \textbf{JobService}: estructura descripciones de trabajo a JSON.
  \item \textbf{RecommendationService}: invoca el motor de recomendación y produce \emph{score}, desglose y resultados completos.
\end{itemize}

\subsection*{Repositorios y persistencia}
Los repositorios (archivo \texttt{api/repositories.py}) encapsulan operaciones CRUD sobre los modelos definidos en \texttt{api/database.py} (motor SQLite). Los modelos son \texttt{CV}, \texttt{JobDescription} y \texttt{Analysis}.

\subsection*{Motor de recomendación}
El motor (archivo \texttt{main/recommendation\_engine.py}) delega en \texttt{ComparatorMain} y hace uso de (\texttt{algoritmo\_recomendacion/comparator\_main.py}) la ejecución de comparadores por aspecto y el cálculo del \emph{score} final (pesos por defecto). Devuelve:
\begin{itemize}
  \item \textbf{score} en \([0, 1]\),
  \item \textbf{score\_breakdown} con contribuciones/ignorados por aspecto,
  \item \textbf{resultado\_completo} con \texttt{comparison\_results} y \texttt{final\_score\_data}.
\end{itemize}

\section{Módulos principales}
\begin{itemize}
  \item \textbf{Limpieza y extracción}: PDF \textrightarrow{} texto.
  \item \textbf{Estructuración}: texto \textrightarrow{} JSON (CVs y Jobs).
  \item \textbf{Comparadores por aspecto}: experiencia, habilidades técnicas, educación, responsabilidades, certificaciones, habilidades blandas, idiomas, ubicación.
  \item \textbf{Cálculo de \emph{score}}: combinación ponderada y normalización si hay aspectos no evaluados.
\end{itemize}

\section{Flujos principales}
\subsection*{1) Subir CV (PDF)}
\noindent Referencia: véase Figura \ref{fig:ui-subir-cv}.
\begin{enumerate}
  \item \texttt{POST /cvs}: la API recibe el archivo.
  \item \textbf{CVService} extrae/limpia texto y estructura el CV.
  \item \textbf{CVRepository} persiste el CV estructurado.
  \item Respuesta: \texttt{cv\_id} y resumen (nombre, email, ubicación).
\end{enumerate}

\subsection*{2) Crear Job (texto)}
\noindent Referencia: véase Figura \ref{fig:ui-crear-job}.
\begin{enumerate}
  \item \texttt{POST /jobs}: la API recibe la descripción.
  \item \textbf{JobService} estructura el texto a JSON.
  \item \textbf{JobRepository} persiste el Job.
  \item Respuesta: \texttt{job\_id} y resumen (título, empresa, ubicación).
\end{enumerate}

\subsection*{3) Analizar CV vs Job}
\noindent Referencia: véase Figura \ref{fig:ui-analizar}.
\begin{enumerate}
  \item \texttt{POST /analyze/\{cv\_id\}/\{job\_id\}}: la API recibe ids.
  \item \textbf{RecommendationService} obtiene CV/Job y ejecuta comparadores.
  \item Se calcula \emph{score}, desglose por aspecto y se persiste el análisis.
  \item Respuesta: \emph{score}, desglose, resumen y tiempo de procesamiento.
\end{enumerate}

\subsection*{4) Ranking de candidatos}
\noindent Referencia: véase Figura \ref{fig:ui-ranking}.
\begin{enumerate}
  \item \texttt{GET /jobs/\{job\_id\}/top-candidatos}.
  \item \textbf{AnalysisRepository} ordena por \emph{score} y retorna el top.
\end{enumerate}

\section{Esquema de datos}
\subsection*{Tabla \texttt{cvs}}
\begin{itemize}
  \item \texttt{id}, \texttt{nombre}, \texttt{email}, \texttt{telefono}, \texttt{ubicacion}, \texttt{cv\_data (JSON)}, \texttt{created\_at}.
\end{itemize}
\subsection*{Tabla \texttt{jobs}}
\begin{itemize}
  \item \texttt{id}, \texttt{titulo}, \texttt{empresa}, \texttt{ubicacion}, \texttt{job\_data (JSON)}, \texttt{created\_at}.
\end{itemize}
\subsection*{Tabla \texttt{analyses}}
\begin{itemize}
  \item \texttt{id}, \texttt{cv\_id}, \texttt{job\_id}, \texttt{nombre\_candidato}, \texttt{titulo\_trabajo}, \texttt{score}, \texttt{score\_breakdown (JSON)}, \texttt{resultado\_completo (JSON)}, \texttt{processing\_time}, \texttt{created\_at}.
\end{itemize}

\section{Endpoints y responsabilidades}
\begin{itemize}
  \item \texttt{POST/GET/GET\{\id\}/DELETE} \texttt{/cvs}, \texttt{/jobs}.
  \item \texttt{POST /analyze/\{cv\_id\}/\{job\_id\}}, \texttt{GET /analyses}, \texttt{GET /analyses/\{id\}}, \texttt{DELETE /analyses/\{id\}}.
  \item \texttt{GET /jobs/\{job\_id\}/top-candidatos}, \texttt{GET /stats}, \texttt{GET /health}.
\end{itemize}
Cada endpoint invoca un \emph{Servicio} y persiste/consulta vía \emph{Repositorio}.

\section{Frontend}
La interfaz (React + Vite + Tailwind) organiza vistas y navegación: \textbf{Dashboard}, \textbf{CVs} (lista, búsqueda, subida, detalle), \textbf{Jobs} (lista, creación, detalle) y \textbf{Analysis} (ejecución, detalle, historial). Los \emph{services} HTTP consumen la API y los componentes presentan resúmenes, breakdowns y rankings.

\section{Decisiones de Diseño}
Principales racionales y \textit{trade-offs}:
\begin{itemize}
  \item \textbf{SQLite}: simplicidad y reproducibilidad en prototipo; para escalar se sugiere un motor administrado.
  \item \textbf{JSON en BD}: flexibilidad para CV/Job y resultados; validación a nivel de aplicación.
  \item \textbf{Comparadores por aspecto}: explicabilidad y control fino; pesos por defecto en la evaluación.
  \item \textbf{CORS en desarrollo}: apertura para pruebas locales; restringir en producción.
  \item \textbf{LLM para estructuración}: rapidez de desarrollo; controlar variabilidad con configuración estable.
\end{itemize}

\section{Seguridad, privacidad y mantenibilidad}
\begin{itemize}
  \item \textbf{Privacidad}: datos personales en HVs; se recomienda consentimiento informado y almacenamiento seguro.
  \item \textbf{Mantenibilidad}: separación en capas, contratos claros y documentación de endpoints y prompts.
  \item \textbf{Escalabilidad}: reemplazo de \texttt{SQLite}, colas para tareas pesadas y caché de resultados.
\end{itemize}
