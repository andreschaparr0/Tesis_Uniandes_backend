% Tesis - Universidad de los Andes
% Proyecto LaTeX base
\documentclass[12pt,letterpaper,oneside]{report}

% ======= Paquetes =======
\usepackage[utf8]{inputenc}
\usepackage[T1]{fontenc}
\usepackage[spanish,es-nodecimaldot]{babel}
\usepackage{wrapfig}
\usepackage{caption}
\usepackage{csquotes}
\usepackage{setspace}
\usepackage{geometry}
\geometry{letterpaper,margin=1in}
\usepackage{graphicx}
\usepackage{float}
\usepackage{booktabs}
\usepackage{multirow}
\usepackage{amsmath}
\usepackage{url}
\usepackage{tcolorbox}
\usepackage{hyperref}
\hypersetup{colorlinks=true,linkcolor=blue,citecolor=blue,urlcolor=blue}

% Bibliografía (biblatex)
\usepackage[backend=biber,style=ieee,maxbibnames=6]{biblatex}
\addbibresource{bibliography.bib}

% Espaciado
\onehalfspacing

% ======= Datos de la tesis =======
\title{Desarrollo y Evaluación de un Algoritmo de Recomendación de Hojas de Vida con NLP para Mejorar la Eficiencia en Reclutamiento}
\author{Andrés Felipe Chaparro Díaz}
\date{Universidad de los Andes\\Facultad de Ingeniería\\Programas: Ingeniería Industrial e Ingeniería de Sistemas y Computación\\2025}

% ======= Documento =======
\begin{document}

% Portada simple
\begin{titlepage}
    \centering
    {\Large Universidad de los Andes\\Facultad de Ingeniería\\Programas: Ingeniería Industrial e Ingeniería de Sistemas y Computación\\}\vspace{1.5cm}
    {\LARGE \textbf{\MakeUppercase{Desarrollo y Evaluación de un Algoritmo de Recomendación de Hojas de Vida con NLP para Mejorar la Eficiencia en Reclutamiento}}\\}\vspace{1.5cm}
    {\large Tesis de grado}\vspace{1cm}

    {\large Autor: Andrés Felipe Chaparro Díaz\\}
    {\large Asesores: Juan Fernando Pérez Bernal (Ingeniería Industrial)\\
    Rubén Francisco Manrique (Ingeniería de Sistemas y Computación)}\\[1.5cm]

    {\large Bogotá, Colombia}\vfill
    {\large 2025}
\end{titlepage}

\pagenumbering{roman}
\tableofcontents
\listoffigures
\listoftables

\chapter*{Resumen}

La gestión eficiente del talento es un desafío crítico, especialmente en el sector tecnológico donde la revisión manual de hojas de vida se convierte en un cuello de botella. Esta tesis presenta el desarrollo y evaluación de un sistema de recomendación de talento basado en Grandes Modelos de Lenguaje (LLMs) diseñado para optimizar el primer filtro de selección. El sistema utiliza técnicas de prompt engineering con modelos GPT para estructurar información de hojas de vida en formato PDF y descripciones de trabajo, permitiendo una comparación semántica profunda en ocho aspectos clave, incluyendo experiencia, habilidades técnicas y educación.

La metodología de evaluación consistió en dos fases experimentales utilizando un conjunto de datos controlado de 10 hojas de vida y 4 vacantes reales del sector TI. Primero, se validó la calidad de la estructuración de datos, alcanzando una precisión superior al 90\% en la extracción de información. Posteriormente, se contrastó el desempeño del motor de recomendación frente al criterio de un experto en Recursos Humanos mediante cuatro escenarios de selección simulados.

Los resultados demostraron que el sistema logra identificar correctamente a los candidatos más idóneos, coincidiendo con el experto humano en el primer lugar del ranking en el 75\% de los casos. Si bien se identificaron limitaciones en la interpretación de reglas de negocio implícitas y contextos geográficos complejos, la herramienta redujo significativamente los tiempos de procesamiento, pasando de aproximadamente 30 minutos manuales a menos de 2 minutos para analizar cinco candidatos. Se concluye que la solución es viable como herramienta de apoyo a la decisión, destacando por su capacidad de generar justificaciones explicables que aportan transparencia y agilidad al proceso de reclutamiento.

\chapter*{Abstract}

Efficient talent management is a critical challenge, especially in the technology sector where manual resume screening becomes a bottleneck. This thesis presents the development and evaluation of a talent recommendation system based on Large Language Models (LLMs) designed to optimize the initial selection filter. The system uses prompt engineering techniques with GPT models to structure information from PDF resumes and job descriptions, enabling deep semantic comparison across eight key aspects, including experience, technical skills, and education.

The evaluation methodology consisted of two experimental phases using a controlled dataset of 10 resumes and 4 real job vacancies from the IT sector. First, data structuring quality was validated, achieving over 90\% accuracy in information extraction. Subsequently, the recommendation engine's performance was contrasted against a Human Resources expert's criteria through four simulated selection scenarios.

Results demonstrated that the system successfully identifies the most suitable candidates, matching the human expert's first-place ranking in 75\% of cases. While limitations were identified in interpreting implicit business rules and complex geographical contexts, the tool significantly reduced processing times, from approximately 30 minutes manually to under 2 minutes for analyzing five candidates. It is concluded that the solution is viable as a decision support tool, standing out for its ability to generate explainable justifications that bring transparency and agility to the recruitment process.


\clearpage
\pagenumbering{arabic}

% ======= Capítulos =======
% Hacer que compile aunque no existan archivos externos (Overleaf)
\makeatletter
\IfFileExists{chapters/01-introduccion.tex}{\chapter{Introducción}
\label{chap:introduccion}

% Contexto y motivación
La gestión eficiente del talento requiere herramientas que permitan relacionar perfiles de candidatos con descripciones de trabajo de forma objetiva, transparente y reproducible. En la práctica, los equipos de reclutamiento enfrentan la revisión manual de grandes volúmenes de hojas de vida (HVs), con formatos heterogéneos (PDFs, imágenes, estilos no estandarizados) y calidad variable de contenido. Este proceso es intensivo en tiempo y recursos, puede introducir sesgos no intencionales en la preselección y dificulta priorizar candidatos óptimos con oportunidad. Este trabajo se enmarca en el nicho de tecnología: el conjunto de HVs utilizado corresponde a perfiles de ingeniería y las descripciones de cargo evaluadas están orientadas a roles de TI y desarrollo de software.

Esta tesis aborda el diseño, implementación y evaluación de un sistema de recomendación que analiza HVs y descripciones de trabajo empleando técnicas de procesamiento de lenguaje natural (NLP) y un esquema de comparación por aspectos. El sistema está compuesto por un backend que procesa y analiza la información, y un frontend que proporciona una interfaz web intuitiva para la gestión de HVs, ofertas de trabajo y visualización de resultados. El sistema toma como entradas una HV (en formato PDF) y una descripción de puesto, extrae y limpia el texto, estructura la información relevante mediante modelos de lenguaje y calcula un \emph{puntaje de ajuste} por distintos aspectos con pesos configurables para reflejar prioridades de negocio. El objetivo es apoyar una preselección inicial más eficiente, objetiva y escalable.

\section{Planteamiento del problema}
\subsection{Situación actual}
La revisión manual de HVs consume una cantidad significativa de tiempo del equipo de reclutamiento y obstaculiza la identificación rápida de candidatos con alto ajuste al puesto. La heterogeneidad de formatos (incluyendo PDFs no indexables o con bajo nivel de estructuración) y la falta de estándares dificulta la automatización y el análisis a escala. Además, la intervención humana temprana puede introducir variabilidad y sesgos no deseados.

\subsection{Situación deseada}
Se busca un prototipo funcional que:
\begin{itemize}
  \item Procese HVs en PDF, extraiga y limpie el texto de forma robusta.
  \item Estructure la información de HVs y descripciones.
  \item Calcule un \emph{puntaje de ajuste} HV--puesto y genere un ranking de candidatos.
  \item Reduzca el tiempo de revisión manual inicial  y aumente la probabilidad de identificar candidatos adecuados en etapas tempranas.
\end{itemize}

\subsection{Restricciones y enfoque}
El desarrollo se realizará con un conjunto de datos inicial de 10 hojas de vida, donde la mayoría corresponden a estudiantes o egresados de la Universidad de los Andes. De estas 10 HVs, 5 pertenecen a estudiantes activos (4 de pregrado y 1 de postgrado). Este conjunto acotado permite adoptar un enfoque iterativo de mejora continua, privilegiando una arquitectura modular, reproducible y auditable que permita refinar comparadores y pesos sin afectar la estabilidad del sistema. La limitación en el tamaño del dataset refleja el alcance inicial del prototipo y permite una evaluación controlada antes de escalar a volúmenes mayores. Adicionalmente, el estudio está acotado al dominio de tecnología; por lo tanto, la generalización de resultados hacia otros sectores (p. ej., salud, finanzas, manufactura) se discute como trabajo futuro y amenaza a la validez externa.

Como refuerzo a la evaluación técnica y para incorporar criterios de negocio propios del dominio, se contó con la participación de un profesional de recursos humanos, Juan Pablo Chaparro (25 años, alrededor de 2 años de experiencia en RR. HH.), quien colaboró en la revisión de resultados y en la interpretación de métricas del sistema. Esta retroalimentación experta contribuyó a contrastar los puntajes generados por el algoritmo con criterios prácticos de selección y a identificar casos límite relevantes para el flujo de reclutamiento.

\section{Pregunta de investigación}
¿En qué medida un sistema de evaluación de hojas de vida basado en LLMs logra alinearse con la clasificación de expertos humanos y reducir los tiempos de revisión en procesos de selección del sector tecnológico?

\section{Objetivos}
\subsection{Objetivo general}
Diseñar, implementar y evaluar un sistema de recomendación de hojas de vida basado en Grandes Modelos de Lenguaje (LLMs), con el fin de mejorar la eficiencia y efectividad del proceso de selección de talento en una empresa.

\subsection{Objetivos específicos}
\begin{enumerate}
  \item Desarrollar una metodología y herramientas para la extracción y estructuración de información a partir de HVs en formato PDF no indexable o poco estructurado.
  \item Diseñar e implementar comparadores por aspecto y un esquema de ponderación configurable para calcular el porcentaje de recomendación.
  \item Evaluar el rendimiento técnico del algoritmo y el impacto potencial del sistema en métricas operacionales de eficiencia y efectividad.
  \item Proponer lineamientos para la integración del sistema en el flujo de trabajo del equipo de reclutamiento.
\end{enumerate}

% Contribuciones
\section{Contribuciones}
\begin{itemize}
  \item Una arquitectura en capas (API, servicios, repositorios y base de datos) que soporta ingestión, análisis y persistencia de resultados, basada en \texttt{FastAPI}, \texttt{SQLAlchemy} y \texttt{SQLite}.
  \item Una interfaz web (frontend) desarrollada en \texttt{React} con \texttt{Vite} y \texttt{TailwindCSS} que permite gestionar HVs, crear ofertas de trabajo, ejecutar análisis de compatibilidad y visualizar resultados mediante gráficos interactivos.
  \item Un módulo de limpieza y extracción de texto de PDFs, y un módulo de estructuración de HVs y descripciones apoyado en orquestación de prompts utilizando \texttt{GPT-4o-mini} de Azure OpenAI.
  \item Un motor de comparación por aspectos con desglose explicativo de puntajes y esquema de pesos configurable.
  \item Un protocolo experimental y un conjunto de métricas para evaluar desempeño técnico y operacional del sistema.
  \item Un componente de validación experta con participación de un profesional de RR. HH. (Juan Pablo Chaparro), para contrastar resultados y ajustar criterios de interpretación.
\end{itemize}

\section{Consideraciones éticas y normativas}
El manejo de datos personales presentes en HVs exige medidas de seguridad, consentimiento informado y cumplimiento de la normativa de protección de datos aplicable. Asimismo, se abordará la mitigación de sesgos algorítmicos mediante transparencia de criterios, explicación de puntajes por aspecto y validaciones con casos de prueba controlados. Estas consideraciones se retoman en detalle al discutir amenazas a la validez y aspectos éticos del estudio.


}{\chapter{Introducción} Plantilla de capítulo.}
\IfFileExists{chapters/02-marco-teorico.tex}{\chapter{Marco Teórico}
\label{chap:marco-teorico}

Este capítulo presenta los fundamentos teóricos que sustentan el desarrollo del sistema de recomendación de hojas de vida. Se abordan conceptos clave para el desarrollo de la tesis principalmente en el procesamiento de lenguaje natural y en los modelos de lenguaje y extracción de información.

\section{Procesamiento de Lenguaje Natural}

El procesamiento de lenguaje natural (NLP) es una disciplina que combina lingüística, informática e inteligencia artificial para dotar a las máquinas de la capacidad de entender, interpretar y generar lenguaje humano \cite{jurafsky2023speech}. En el contexto de este trabajo, las técnicas de NLP son fundamentales para transformar documentos no estructurados (HVs en formato PDF) en representaciones estructuradas que permitan análisis automatizado.

\subsection{Extracción y Limpieza de Texto}

La extracción y limpieza de texto son procesos fundamentales en el análisis automático de documentos. La extracción se refiere a recuperar el contenido textual de un archivo, el cual puede encontrarse estructurado o representar texto a través de imágenes que requieren técnicas de reconocimiento óptico de caracteres. Una vez obtenido el contenido, la etapa de limpieza busca normalizarlo y eliminar inconsistencias propias de la digitalización o del formato, como caracteres irregulares, fragmentación de palabras o elementos considerados ruido. Estas transformaciones permiten obtener un texto coherente y uniforme que pueda ser utilizado en posteriores tareas de procesamiento y análisis.

\subsection{Pipeline de procesamiento}

 El término \emph{pipeline} se utiliza para describir la cadena de etapas que transforma entradas crudas que pueden estar en diferentes formatos (p.\,ej., imágenes en PDF o texto plano), en salidas útiles para la toma de decisiones. En sistemas de procesamiento de lenguaje natural y recomendación, un pipeline típico comienza con la extracción y limpieza de texto, continúa con la estructuración de la información (p.\,ej., conversión de texto libre a representaciones con campos definidos) y culmina en algún tipo de análisis o comparación (clasificación, ranking, cálculo de scores, etc.). Pensar el sistema como un pipeline permite razonar sobre cada etapa de forma modular---qué insumos recibe, qué transforma y qué entrega a la siguiente capa---y facilita tanto la depuración (identificar en qué punto se introducen errores) como la reproducibilidad (repetir el mismo flujo sobre nuevos datos manteniendo configuraciones constantes).


\section{Modelos de Lenguaje y Extracción de Información}

Esta sección describe cómo los modelos de lenguaje y las prácticas de diseño de \emph{prompts} permiten transformar texto libre en representaciones estructuradas útiles para el sistema: primero se presentan las capacidades de los LLMs para comprender contexto y semántica; luego se explica cómo, mediante \emph{prompt engineering} y un framework de python, se obtienen salidas en formatos controlados (p.\,ej., JSON) que alimentan los módulos de estructuración y comparaciónn.

\subsection{Modelos de Lenguaje Grandes (LLMs)}

Los modelos de lenguaje grandes son sistemas de aprendizaje profundo entrenados con grandes volúmenes de texto, capaces de aprender representaciones complejas del lenguaje y comprender su contexto y semántica. En este trabajo se emplea GPT-4o-mini, un modelo de la familia GPT (Generative Pre-trained Transformer) desarrollado por OpenAI y desplegado a través de Azure OpenAI Services. Estos modelos pueden generar respuestas estructuradas a partir de instrucciones bien definidas y adaptarse a distintos dominios o tareas con un ajuste mínimo.


\subsection{Prompt Engineering}

El \emph{prompt engineering} consiste en diseñar instrucciones que guían a los modelos de lenguaje para generar resultados precisos y estructurados. En tareas de extracción y organización de información, esta práctica permite definir con claridad el formato de salida y descomponer procesos complejos en pasos intermedios. Al incluir ejemplos dentro del propio prompt, el modelo puede adaptarse mejor al contexto y producir respuestas más consistentes. Para la orquestación de estas interacciones se emplea LangChain \cite{langchain}, un framework que permite gestionar plantillas de prompts, estructurar las respuestas mediante parsers, manejar errores y conectar distintos proveedores de modelos.




\subsection{Estrategias de Prompting}

Existen diversas técnicas para diseñar prompts efectivos dependiendo de la tarea a realizar. A continuación se describen las estrategias fundamentales utilizadas en sistemas de extracción de información:

\subsubsection{Role Prompting (Asignación de Roles)}
Esta técnica consiste en asignar una identidad o función específica al modelo al inicio de la interacción (p.\,ej., "Eres un experto en recursos humanos"). Al definir un rol, se condiciona el tono, el vocabulario y el enfoque de la respuesta, alineando la generación del modelo con las expectativas del dominio específico de la tarea \cite{jurafsky2023speech}.

\subsubsection{Zero-Shot Prompting}
En el \emph{zero-shot prompting}, se presenta al modelo una tarea sin proveer ejemplos previos de la solución esperada. El modelo debe inferir cómo resolver el problema basándose únicamente en su entrenamiento previo y en la descripción textual de la instrucción. Esta estrategia es efectiva con modelos de gran capacidad (LLMs) que han generalizado conocimiento sobre múltiples tareas durante su fase de entrenamiento.

\subsubsection{Structured Output Prompting (Salida Estructurada)}
Esta estrategia se enfoca en restringir el formato de la respuesta generada. En lugar de solicitar texto libre, se instruye al modelo para que genere su salida siguiendo un esquema rígido, comúnmente JSON o XML. Esto es crítico para integrar modelos de lenguaje en sistemas de software automatizados, ya que permite que la salida del modelo sea parseada y validada programáticamente sin necesidad de post-procesamiento complejo de lenguaje natural.

\subsection{LangChain}

LangChain es un framework diseñado para facilitar la construcción de aplicaciones basadas en modelos de lenguaje mediante la definición de cadenas de procesamiento, plantillas de prompts reutilizables y mecanismos de estandarización de entradas y salidas. Además, incorpora herramientas para validar formatos de salida, integrar proveedores externos de modelos y gestionar parámetros de ejecución de forma consistente. Estas características lo convierten en una plataforma adecuada para desarrollar sistemas que requieran orquestar interacciones complejas con modelos de lenguaje de manera reproducible y controlada.\cite{langchain}

}{\chapter{Marco Teórico} Contenido placeholder.}
\IfFileExists{chapters/03-estado-del-arte.tex}{\chapter{Estado del Arte}
\label{chap:estado-del-arte}

\section{Sistemas de Recomendación de Talento}
Revisión de enfoques académicos e industriales para emparejamiento HV--puesto.

\section{Extracción y Estructuración de HVs}
Técnicas, esquemas de información y retos en documentos no estructurados.

\section{Evaluación y Métricas}
Prácticas de evaluación, datasets y métricas (precisión, ranking, correlación, etc.).


}{\chapter{Estado del Arte} Contenido placeholder.}
\IfFileExists{chapters/04-metodologia.tex}{\chapter{Metodología}
\label{chap:metodologia}

\section{Preguntas de Investigación}

Este trabajo busca responder a la pregunta principal planteada en la Introducción: ¿Cómo el diseño e implementación de un sistema de recomendación de hojas de vida basado en procesamiento de lenguaje natural mejora la efectividad del proceso de selección de talento?

Derivamos las siguientes preguntas específicas:
\begin{enumerate}
  \item ¿Con qué calidad se estructuran las hojas de vida y las descripciones de trabajo (por rubro/campo) al transformarlas a JSON?
  \item ¿En qué medida el \emph{score} de compatibilidad del sistema y sus explicaciones por aspecto son considerados adecuados por un profesional de RR.\,HH.?
  \item ¿En qué medida coinciden los rankings de candidatos producidos por el sistema con el ranking emitido por un profesional de RR.\,HH.?
\end{enumerate}

Hipótesis asociadas:
\begin{itemize}
  \item \textbf{H1 (Estructuración CV/Job)}: La estructuración automática alcanza puntajes altos (0--100) por rubro al compararse con las fuentes originales (PDF del CV y texto del Job).
  \item \textbf{H2 (Adecuación de score y explicación)}: Un profesional de RR.\,HH. califica con valores altos (0--100) tanto el \emph{score} del sistema como las razones por aspecto (experiencia, responsabilidades, habilidades, etc.).
  \item \textbf{H3 (Consistencia de ranking)}: El orden de candidatos del sistema muestra coincidencias relevantes con el ranking emitido por el profesional de RR.\,HH.
\end{itemize}

\section{Diseño Experimental}

\subsection{Contexto y dataset}
Se emplea un conjunto acotado de 10 hojas de vida (la mayoría de estudiantes/egresados de la Universidad de los Andes; 5 estudiantes activos) y descripciones de cargos en tecnología (desarrollo de software y TI). Este tamaño permite un ciclo iterativo de prueba y mejora controlado.

\subsection{Variables}
\textbf{Independientes}:
\begin{itemize}
  \item Descripción de cargo (tarea) frente a la cual se evalúan candidatos.
  \item Conjunto de hojas de vida analizadas por tarea.
\end{itemize}
\textbf{Dependientes}:
\begin{itemize}
  \item Calidad de estructuración (0--100) por rubro/campo en CVs y Jobs.
  \item Calificación (0--100) del \emph{score} del sistema por análisis.
  \item Calificación (0--100) de las razones/explicaciones por aspecto.
  \item Comparación de rankings entre experto y sistema (coincidencias/diferencias).
\end{itemize}

\subsection{Métricas}
\begin{itemize}
  \item \textbf{Estructuración de CVs y Jobs}: Puntuación 0--100 por rubro al contrastar JSON contra el PDF (CV) o texto (Job). Se reportan promedios por rubro y un promedio global por documento.
  \item \textbf{Adecuación del modelo (por análisis)}:
    \begin{itemize}
      \item Calificación 0--100 del \emph{score} de compatibilidad producido por el sistema.
      \item Calificación 0--100 de la explicación por aspecto (experiencia, responsabilidades, habilidades técnicas, habilidades blandas, certificaciones, idiomas, ubicación).
    \end{itemize}
  \item \textbf{Ranking experto vs.\ sistema}: Se comparan las listas ordenadas de candidatos por análisis y se registran coincidencias y diferencias principales (sin aplicar métricas de precisión/recuperación).
\end{itemize}

\subsection{Procedimiento}
\begin{enumerate}
  \item \textbf{Preparación}: Limpieza y estructuración de HVs y descripciones (pipeline de extracción; ver capítulos previos). Registro de versiones de prompts y configuraciones.
  \item \textbf{Análisis automático}: Para cada combinación \emph{(CV, Job)} se ejecuta el análisis con pesos \emph{default}. Se almacenan \emph{score}, desglose por aspecto y tiempos.
  \item \textbf{Estructuración (validación manual)}: Para CVs y Jobs se revisa rubro por rubro el JSON estructurado contra la fuente (PDF o texto) y se asigna una calificación 0--100 por rubro. Se agregan promedios por rubro y globales por documento.
  \item \textbf{Evaluación experta del modelo}: Un profesional de RR.\,HH., Juan Pablo Chaparro (25 años, \(\sim\)2 años de experiencia), realiza 4 análisis; en cada uno evalúa 5 hojas de vida (20 evaluaciones en total). Para cada análisis:
    \begin{itemize}
      \item Califica 0--100 el \emph{score} de compatibilidad entregado por el sistema.
      \item Califica 0--100 las razones por aspecto (experiencia, responsabilidades, habilidades, certificaciones, idiomas, ubicación).
      \item Emite un ranking de candidatos y se compara cualitativamente con el ranking del sistema (coincidencias/diferencias).
    \end{itemize}
\end{enumerate}

\subsection{Amenazas y controles}
\begin{itemize}
  \item \textbf{Tamaño muestral reducido}: Se reportan resultados con intervalos y se evita sobreinterpretación; se propone ampliar dataset como trabajo futuro.
  \item \textbf{Variabilidad del modelo de lenguaje}: Se fija configuración conservadora (temperatura baja) y se almacenan salidas intermedias para trazabilidad.
  \item \textbf{Sesgo de dominio}: El estudio se acota a roles de tecnología; se declara la validez externa como amenaza y se discute su impacto.
\end{itemize}

\section{Reproducibilidad}

\subsection{Entorno y dependencias}
Se utiliza Python 3.11, FastAPI, SQLAlchem}, PyMuPDF, NLTK, LangChain y cliente de Azure OpenAI. Las versiones se especifican en \texttt{requirements.txt}. 

\subsection{Modelos y configuración}
Se emplea \texttt{GPT-4o-mini} vía Azure OpenAI con variables de entorno (\texttt{.env}). Se recomienda fijar temperatura baja y registrar \emph{prompts} en los apéndices. Las respuestas estructuradas (CVs/Jobs) y resultados de análisis se persisten en \texttt{cv\_system.db}.

\subsection{Datos y trazabilidad}
Los PDFs de HVs y textos de descripciones se versionan como artefactos de datos (o referencias anonimizadas). Se guarda:
\begin{itemize}
  \item Texto extraído y versiones de limpieza.
  \item JSON estructurado por entidad (CV/Job).
  \item Resultados de análisis: \emph{score}, desglose por aspecto, tiempos.
  \item Anotaciones/orden del experto de RR.\,HH.
\end{itemize}




}{\chapter{Metodología} Contenido placeholder.}
\IfFileExists{chapters/05-diseno-implementacion.tex}{\chapter{Diseño, Arquitectura e Implementación}
\label{chap:diseno-implementacion}

% Capítulo unificado (contenido a desarrollar)


}{\chapter{Diseño, Arquitectura e Implementación} Contenido placeholder.}
\IfFileExists{chapters/06-experimentos-resultados.tex}{\chapter{Evaluación y Resultados}
\label{chap:evaluaciones}

\section{Diseño experimental}
\label{sec:diseno-experimental}

Esta sección detalla el diseño de las pruebas realizadas para validar la utilidad y precisión del sistema propuesto. El protocolo experimental se construyó para evaluar tanto la capacidad del sistema para interpretar información no estructurada como su competencia para emitir juicios de valor comparables a los de un reclutador humano.

\subsection{Visión general del protocolo de experimentación}

El proceso de validación se estructuró en torno a dos ejes fundamentales definidos en la metodología. En primer lugar, se evaluó la calidad de la \textbf{estructuración de datos}, verificando si los modelos de lenguaje lograban extraer y normalizar correctamente la información contenida en las hojas de vida (PDF) y las descripciones de trabajo (texto). Esta etapa es crítica, pues la fiabilidad de cualquier recomendación posterior depende directamente de la integridad de los datos de entrada.

En segundo lugar, se evaluó el desempeño del \textbf{motor de recomendación}. Para esto, se diseñó un escenario de prueba que simula procesos de selección reales, donde el sistema procesa un conjunto de candidatos frente a vacantes específicas. Los resultados generados (scores de compatibilidad, desgloses por aspecto y rankings) fueron sometidos a una revisión cualitativa por parte de un profesional de Recursos Humanos, quien contrastó el criterio de la inteligencia artificial con su propio juicio experto. Este enfoque mixto permite medir no solo la precisión técnica, sino la utilidad práctica de la herramienta como sistema de apoyo a la decisión.

\subsection{Delimitación del dominio: El nicho tecnológico}

Para garantizar la validez interna de las evaluaciones y permitir un análisis profundo de los resultados, el alcance del modelo se acotó específicamente al sector de tecnología y desarrollo de software. Esta decisión se fundamenta en el perfil académico del autor (Ingeniería de Sistemas y Computación), lo que otorga un criterio técnico sólido para validar si las correspondencias halladas entre habilidades (por ejemplo, frameworks, lenguajes de programación y herramientas) son semánticamente correctas.

\subsection{Conformación del dataset}

El conjunto de datos utilizado para las pruebas está compuesto por un total de 10 hojas de vida y 4 descripciones de trabajo, todas pertenecientes al sector tecnológico. Las hojas de vida fueron recolectadas mediante una convocatoria controlada entre conocidos y colegas del entorno académico, quienes autorizaron el uso de sus documentos con fines académicos.

La muestra de candidatos refleja el contexto universitario y de ingreso al mercado laboral, con una mayoría de participantes vinculados a la Universidad de los Andes. En términos de experiencia, se obtuvo una distribución equilibrada:
\begin{itemize}
    \item Cerca del 40\% corresponde a estudiantes de últimos semestres o recién egresados, cuyos perfiles se basan principalmente en proyectos académicos y prácticas profesionales.
    \item El 60\% restante corresponde a profesionales con experiencia que oscila entre perfiles junior y candidatos con hasta seis años de trayectoria en la industria.
\end{itemize}

Las 4 descripciones de trabajo se seleccionaron para cubrir distintos niveles dentro del desarrollo de software, buscando variedad en responsabilidades y en el nivel de experiencia solicitado. Para cada una de las dos empresas consideradas, se incluyó tanto una vacante orientada a candidatos con experiencia profesional como una oportunidad de prácticas dirigida a perfiles en formación. Esta combinación permitió evaluar cómo responde el modelo frente a ofertas que difieren en complejidad, nivel de detalle y expectativas técnicas, sin introducir sesgos hacia un único tipo de rol.

\section{Evaluación de estructuración}
\label{sec:evaluacion-estructuracion}

\subsection{Metodología de verificación técnica}
Esta fase de evaluación tiene como objetivo medir la fidelidad con la que el sistema transforma documentos no estructurados en objetos JSON. La evaluación de la estructuración es una tarea de verificación fáctica: comprobar si la información presente en el documento original fue capturada correctamente, si hubo omisiones o si el modelo alucinó datos inexistentes. Por esto mismo, esta etapa la evaluó una persona con conocimiento del contexto del proyecto (sin requerir especialidad en RR. HH.). 

Se procesaron las 10 hojas de vida y las 4 descripciones de trabajo a través de los pipelines de estructuración descritos en el capítulo anterior. Posteriormente, se realizó una revisión manual ítem por ítem, comparando el contenido del archivo fuente (PDF o texto) contra el resultado estructurado generado por el sistema.

\subsection{Instrumentos y métricas de calidad}
Para evaluar de forma consistente la calidad de la estructuración, se definieron dos instrumentos análogos, uno para hojas de vida y otro para descripciones de trabajo. En ambos casos, cada campo extraído se somete a una revisión manual donde se asigna un \textit{Score} porcentual (de 0\% a 100\%) que mide la completitud y precisión de la extracción, acompañado de una \textit{Razón} justificativa que explica el motivo del puntaje.

Para las hojas de vida, el instrumento evalúa siete dimensiones clave: información personal, educación, experiencia laboral, habilidades técnicas, habilidades blandas, certificaciones e idiomas. Por su parte, el instrumento para descripciones de trabajo amplía la evaluación a diez dimensiones, incorporando campos específicos de las vacantes como información básica, responsabilidades, ubicación y beneficios, además de los requisitos homólogos a los del CV (educación, experiencia, habilidades y certificaciones).

\subsection{Resultados de estructuración de hojas de vida}

Tras aplicar el instrumento de evaluación sobre las 10 hojas de vida, se obtuvieron un total de 70 resultados (7 rubros por cada uno de los 10 candidatos). El desempeño general del motor de estructuración fue altamente satisfactorio. A continuación, se presenta la distribución de los puntajes obtenidos:

\begin{table}[H]
    \centering
    \small
    \caption{Distribución de puntajes de estructuración de Hojas de Vida}
    \label{tab:distribucion-cv}
    \begin{tabular}{|l|c|l|}
    \hline
    \textbf{Rango de Score} & \textbf{Frecuencia} & \textbf{Descripción} \\ \hline
    100\% & 51 & Extracción perfecta \\ \hline
    80\% - 99\% & 12 & Errores menores de formato o incompletitud leve \\ \hline
    50\% - 79\% & 3 & Errores parciales o confusión de categorías \\ \hline
    20\% - 49\% & 1 & Error significativo de interpretación \\ \hline
    0\% - 19\% & 0 & Fallo total \\ \hline
    \end{tabular}
\end{table}

El análisis cualitativo de estos resultados permite identificar fortalezas claras y áreas de mejora específicas en el pipeline de procesamiento:

\subsubsection*{Desempeño por rubros y hallazgos clave}

Las Tablas \ref{tab:resultados-cv-1} y \ref{tab:resultados-cv-2} presentan la matriz completa de evaluación para los 10 candidatos, detallando el score y la razón en cada rubro.

\begin{table}[H]
    \centering
    \scriptsize
    \caption{Resultados de estructuración de Hojas de Vida (Candidatos 1-5)}
    \label{tab:resultados-cv-1}
    \setlength{\tabcolsep}{3pt}
    \begin{tabular}{|l|p{2.2cm}|p{2.2cm}|p{2.2cm}|p{2.2cm}|p{2.2cm}|}
    \hline
    \textbf{Rubro} & \textbf{Andrés Chaparro} & \textbf{Carlos Santoyo} & \textbf{Gabriel Gómez} & \textbf{Javier Barrera} & \textbf{Juan E. López} \\ \hline
    \textbf{Personal} & \textbf{100\%} \newline Bien & \textbf{95\%} \newline Faltaron símbolos en el correo & \textbf{100\%} \newline Bien & \textbf{95\%} \newline Faltaron símbolos en el correo & \textbf{95\%} \newline Faltaron símbolos en el correo \\ \hline
    \textbf{Educación} & \textbf{100\%} \newline Bien & \textbf{100\%} \newline Bien & \textbf{60\%} \newline Puso mal la univ. (Andes vs Javeriana) & \textbf{100\%} \newline Bien & \textbf{100\%} \newline Bien \\ \hline
    \textbf{Experiencia} & \textbf{100\%} \newline Bien & \textbf{80\%} \newline Faltó poner que call centers fueron bilingües & \textbf{35\%} \newline Confundió proy. con exp. y añadió de más & \textbf{100\%} \newline Bien & \textbf{90\%} \newline Faltó especificar clase de monitoría \\ \hline
    \textbf{Tech Skills} & \textbf{100\%} \newline Bien & \textbf{100\%} \newline Bien & \textbf{100\%} \newline Bien & \textbf{100\%} \newline Bien & \textbf{100\%} \newline Bien \\ \hline
    \textbf{Soft Skills} & \textbf{100\%} \newline Bien & \textbf{100\%} \newline Bien & \textbf{100\%} \newline Bien & \textbf{80\%} \newline Bien pero faltaron algunas & \textbf{100\%} \newline Bien \\ \hline
    \textbf{Certif.} & \textbf{100\%} \newline Bien & \textbf{90\%} \newline Puso año 2023 no aclarado en CV & \textbf{100\%} \newline Bien & \textbf{100\%} \newline Bien & \textbf{100\%} \newline Bien \\ \hline
    \textbf{Idiomas} & \textbf{100\%} \newline Bien & \textbf{70\%} \newline Puso el lenguaje bilingüe & \textbf{100\%} \newline Bien & \textbf{100\%} \newline Bien & \textbf{100\%} \newline Bien \\ \hline
    \end{tabular}
\end{table}

\begin{table}[H]
    \centering
    \scriptsize
    \caption{Resultados de estructuración de Hojas de Vida (Candidatos 6-10)}
    \label{tab:resultados-cv-2}
    \setlength{\tabcolsep}{3pt}
    \begin{tabular}{|l|p{2.2cm}|p{2.2cm}|p{2.2cm}|p{2.2cm}|p{2.2cm}|}
    \hline
    \textbf{Rubro} & \textbf{Juanchito Bernal} & \textbf{Julián Galindo} & \textbf{María Ramírez} & \textbf{Stiven Viedman} & \textbf{William Xavier} \\ \hline
    \textbf{Personal} & \textbf{95\%} \newline Faltaron símbolos en el correo & \textbf{95\%} \newline Faltaron símbolos en el correo & \textbf{95\%} \newline Faltaron símbolos en el correo & \textbf{100\%} \newline Bien & \textbf{100\%} \newline Bien \\ \hline
    \textbf{Educación} & \textbf{100\%} \newline Bien & \textbf{100\%} \newline Bien & \textbf{100\%} \newline Bien & \textbf{100\%} \newline Bien & \textbf{100\%} \newline Bien \\ \hline
    \textbf{Experiencia} & \textbf{50\%} \newline Correcta pero añadió proyectos portafolio & \textbf{100\%} \newline Bien & \textbf{100\%} \newline Bien & \textbf{100\%} \newline Bien & \textbf{100\%} \newline Bien \\ \hline
    \textbf{Tech Skills} & \textbf{100\%} \newline Bien & \textbf{100\%} \newline Bien & \textbf{100\%} \newline Bien & \textbf{100\%} \newline Bien & \textbf{50\%} \newline Capta todas pero pone de más \\ \hline
    \textbf{Soft Skills} & \textbf{100\%} \newline Bien & \textbf{100\%} \newline Bien & \textbf{80\%} \newline No puso, se podría intuir & \textbf{80\%} \newline No puso, se podría intuir & \textbf{100\%} \newline Bien \\ \hline
    \textbf{Certif.} & \textbf{100\%} \newline Bien & \textbf{100\%} \newline Bien & \textbf{100\%} \newline Bien & \textbf{100\%} \newline Bien & \textbf{50\%} \newline Puso cursos de educ. como certif. \\ \hline
    \textbf{Idiomas} & \textbf{100\%} \newline Bien & \textbf{100\%} \newline Bien & \textbf{100\%} \newline Bien & \textbf{100\%} \newline Bien & \textbf{70\%} \newline No hay, se puede intuir \\ \hline
    \end{tabular}
\end{table}

\begin{itemize}
    \item \textbf{Habilidades Blandas (Soft Skills):} En la mayoría de los casos, el sistema logró inferir correctamente habilidades implícitas que no estaban listadas textualmente. Sin embargo, en 3 de los 10 casos, el modelo fue excesivamente conservador y no reportó habilidades que, aunque no explícitas, podrían haberse intuido del contexto, resultando en puntajes del 80\%.
    
    \item \textbf{Experiencia Laboral:} Este fue el rubro que presentó los desafíos más complejos. Si bien la mayoría de extracciones fueron correctas (100\%), el modelo mostró una tendencia a alucinar o sobre-interpretar información en perfiles junior. En casos específicos como el de Gabriel Gómez (35\%) o Juanchito Bernal (50\%), el sistema clasificó proyectos académicos de portafolio o voluntariados como experiencia laboral formal. Aunque semánticamente relavante, esto constituye un error estructural que infla la trayectoria del candidato al mezclar experiencia real con académica.
    
    \item \textbf{Información Personal y Educación:} Estos campos mostraron la mayor estabilidad. Los errores encontrados fueron marginales (scores del 95\%), limitándose principalmente a detalles de formato, como la omisión de símbolos en correos electrónicos (observado en 5 candidatos) o, en un único caso, la atribución errónea de la universidad en un título de pregrado.
    
    \item \textbf{Habilidades Técnicas y Certificaciones:} La extracción fue precisa en casi la totalidad de la muestra (generalmente 100\%). Se registró un caso aislado (William Xavier, 50\%) donde el modelo interpretó cursos listados en educación como certificaciones independientes, y otro donde incluyó habilidades técnicas excedentes.
    
    \item \textbf{Idiomas:} El desempeño fue sólido (mayoría 100\%), con excepciones puntuales relacionadas con la inferencia del lenguaje ``bilinguismo'' o la falta de deducción del idioma nativo cuando este no se declaraba explícitamente.

\end{itemize}

En conclusión, el motor de estructuración demostró una alta fiabilidad para los fines del sistema de recomendación, con una tasa de éxito superior al 90\% en la mayoría de los campos. Los errores identificados, principalmente en la distinción entre proyectos y experiencia laboral, son consistentes con la ambigüedad propia de los perfiles junior en el sector tecnológico.

\subsection{Resultados de estructuración de descripciones de trabajo}
La evaluación de la estructuración de las descripciones de trabajo (Jobs) se realizó sobre 4 vacantes distribuidas entre las empresas Pragma y Sezzle. Se obtuvieron un total de 40 puntos de control (10 rubros por vacante). Al igual que con las hojas de vida, los resultados fueron predominantemente positivos, con una alta incidencia de extracciones perfectas.

\begin{table}[H]
    \centering
    \small
    \caption{Distribución de puntajes de estructuración de Descripciones de Trabajo}
    \label{tab:distribucion-jobs}
    \begin{tabular}{|l|c|}
    \hline
    \textbf{Rango de Score} & \textbf{Frecuencia} \\ \hline
    100\% & 30 \\ \hline
    80\% - 99\% & 8 \\ \hline
    50\% - 79\% & 0 \\ \hline
    20\% - 49\% & 1 \\ \hline
    0\% - 19\% & 1 \\ \hline
    \end{tabular}
\end{table}

A continuación, la Tabla \ref{tab:resultados-jobs} presenta el detalle de la evaluación.

\begin{table}[H]
    \centering
    \scriptsize
    \caption{Resultados detallados de estructuración de Descripciones de Trabajo}
    \label{tab:resultados-jobs}
    \setlength{\tabcolsep}{3pt}
    \begin{tabular}{|l|p{3.0cm}|p{3.0cm}|p{3.0cm}|p{3.0cm}|}
    \hline
    \textbf{Rubro} & \textbf{Pragma Dev Java} & \textbf{Pragma Practicante} & \textbf{Sezzle AI Intern} & \textbf{Sezzle Jr Engineer} \\ \hline
    \textbf{Info. Básica} & \textbf{100\%} \newline Bien & \textbf{100\%} \newline Bien & \textbf{100\%} \newline Bien & \textbf{100\%} \newline Bien \\ \hline
    \textbf{Responsab.} & \textbf{100\%} \newline Excelente, extrajo bien de "Retos" & \textbf{100\%} \newline Excelente, extrajo bien de "Retos" & \textbf{100\%} \newline Excelente, extrajo de "What You'll Do" & \textbf{100\%} \newline Excelente, extrajo de "What You'll Do" \\ \hline
    \textbf{Ubicación} & \textbf{100\%} \newline Correctas las 3 ubicaciones & \textbf{100\%} \newline Bien & \textbf{100\%} \newline Bien & \textbf{100\%} \newline Bien \\ \hline
    \textbf{Educación} & \textbf{80\%} \newline Se desconoce pero infiere Comp. Science & \textbf{100\%} \newline Bien & \textbf{100\%} \newline Infiere bien aunque no sea explícito & \textbf{100\%} \newline Infiere bien aunque no sea explícito \\ \hline
    \textbf{Experiencia} & \textbf{90\%} \newline Faltó exp. en caché (Redis/Memcached) & \textbf{80\%} \newline Pone "Desconocido" (correcto) pero mejor decir "no necesario" & \textbf{80\%} \newline Pone "Desconocido" (correcto) pero mejor decir "no necesario" & \textbf{20\%} \newline Solo pone años, no especifica en qué \\ \hline
    \textbf{Tech Skills} & \textbf{100\%} \newline Bien & \textbf{100\%} \newline Bien & \textbf{100\%} \newline Bien & \textbf{100\%} \newline Bien \\ \hline
    \textbf{Soft Skills} & \textbf{100\%} \newline Intuye 3 habilidades con sentido & \textbf{100\%} \newline Intuye 3 habilidades con sentido & \textbf{100\%} \newline Bien & \textbf{100\%} \newline Bien \\ \hline
    \textbf{Certif.} & \textbf{100\%} \newline Bien (Desconocido) & \textbf{100\%} \newline Bien (Desconocido) & \textbf{100\%} \newline Bien & \textbf{100\%} \newline Bien \\ \hline
    \textbf{Idiomas} & \textbf{90\%} \newline Infiere Español intermedio & \textbf{90\%} \newline Infiere Español intermedio & \textbf{90\%} \newline Infiere Inglés & \textbf{90\%} \newline Infiere Inglés \\ \hline
    \textbf{Beneficios} & \textbf{100\%} \newline Bien (Desconocido) & \textbf{10\%} \newline Confunde salario con beneficio & \textbf{100\%} \newline Bien (Desconocido) & \textbf{100\%} \newline Bien (Desconocido) \\ \hline
    \end{tabular}
\end{table}

Los hallazgos principales de esta evaluación indican:
\begin{itemize}
    \item \textbf{Responsabilidades y Ubicación:} Estos fueron los campos con el desempeño más sólido y constante, logrando una precisión del 100\% en todos los casos. El modelo demostró gran capacidad de adaptación semántica, identificando correctamente las secciones equivalentes a responsabilidades independientemente del título utilizado en la oferta.
    
    \item \textbf{Experiencia:} Al igual que en las hojas de vida, este rubro presentó las mayores dificultades. En las vacantes de practicantes, el modelo marcó correctamente la experiencia como Desconocida (score 80\%), aunque una interpretación ideal hubiera explicitado que "no se requiere experiencia". El error más significativo (20\%) ocurrió en la vacante junior de Sezzle, donde el sistema extrajo únicamente la cantidad de años requeridos, omitiendo el contexto técnico crucial sobre en qué tecnologías debía tenerse dicha experiencia.
    
    \item \textbf{Información Básica y Educación:} La extracción fue altamente confiable (mayoría 100\%). Destaca positivamente la capacidad del modelo para inferir requisitos educativos implícitos; por ejemplo, en las ofertas de Sezzle, aunque no se listaba una carrera específica, el sistema dedujo correctamente que el perfil requería estudios en Ciencias de la Computación o afines.
    
    \item \textbf{Habilidades Técnicas, Blandas y Certificaciones:} La precisión fue excelente en estos apartados. En habilidades técnicas y certificaciones no se registraron errores (100\%). En habilidades blandas, el sistema aplicó exitosamente la lógica de inferencia de la sub-seccion \ref{subsec:pipeline-jobs}, deduciendo tres competencias transversales pertinentes para cada cargo a partir de las responsabilidades descritas.
    
    \item \textbf{Idiomas:} Se validó exitosamente la funcionalidad de detección automática de la sub-seccion \ref{subsec:pipeline-jobs}. En los 4 casos, ante la ausencia de un requisito explícito, el modelo dedujo correctamente el idioma base de la oferta (Español para Pragma, Inglés para Sezzle) y asignó un nivel intermedio por defecto, garantizando que el criterio lingüístico no quedara vacío.
    
    \item \textbf{Beneficios:} Aunque en general el desempeño fue correcto, se registró un error puntual grave (10\%) en la vacante de practicante de Pragma. El sistema confundió el salario clasificándolo erróneamente como un beneficio, cuando este elemento salario deberia aparecer en la información basica, lo que revela una dificultad para distinguir matices contractuales específicos de la legislación laboral local.
\end{itemize}
\section{Evaluación del modelo}
\label{sec:evaluacion-modelo}

\subsection{Protocolo de validación experta y métricas comparativas}
\label{subsec:protocolo-validacion}

Para determinar si el sistema emite juicios de valor alineados con el criterio profesional, se diseñó una evaluación basada en cuatro análisis de procesos de selección. Cada análisis consistió en el análisis de una descripción de trabajo específica frente a un subconjunto de cinco hojas de vida seleccionadas del dataset, creando escenarios controlados de competencia por una vacante.

La validación cualitativa fue realizada por Juan Pablo Chaparro, profesional de Recursos Humanos con experiencia en reclutamiento. Su rol consistió en actuar como un juez y, posteriormente, como auditor del sistema. Para cada análisis, el experto analizó los mismos documentos que el algoritmo y generó su propia valoración. Las métricas de éxito se definieron en función de la convergencia entre el criterio humano y el artificial a través de tres dimensiones:

\begin{itemize}
    \item \textbf{Comparación de rankings:} Se contrasta el orden de idoneidad de los candidatos generado por el modelo frente al ranking construido por el evaluador humano. Esto permite medir si el sistema prioriza correctamente a los mejores perfiles.
    \item \textbf{Consistencia de puntajes:} Se evalúa la correlación entre el \textit{score} de compatibilidad calculado por el algoritmo y la valoración subjetiva (en escala 0-100) asignada por el experto.
    \item \textbf{Validación semántica de las razones:} El experto revisa las justificaciones textuales generadas por la IA para cada aspecto (ej. "¿Por qué la experiencia es compatible?"), calificando si la explicación es coherente, precisa y útil para la toma de decisiones.
\end{itemize}

\subsection{Fases de ejecución y refinamiento iterativo}
\label{subsec:fases-evaluacion}

Con el objetivo de asegurar no solo la evaluación sino también la mejora continua del prototipo, el proceso experimental se dividió en dos fases operativas:

\begin{itemize}
    \item \textbf{Fase 1:} Se ejecutó el primer análisis utilizando la vacante Junior Java Developer de la empresa Pragma. Los resultados de este primer análisis fueron sometidos a una sesión de revisión detallada con el experto de RR.\,HH. Esta etapa permitió identificar divergencias tempranas en la interpretación de ciertos criterios (particularmente en la gestión de certificaciones y habilidades), lo que derivó en ajustes puntuales a la lógica de los comparadores y a los pesos del sistema.
    
    \item \textbf{Fase 2:} Tras implementar las correcciones derivadas de la reunión de retroalimentación, se procedió a ejecutar los tres análisis restantes. En esta fase se consolidaron los resultados definitivos reportados en este capítulo, permitiendo medir el desempeño del sistema ya calibrado frente a escenarios de distinta complejidad (roles de practicante vs. roles junior).
\end{itemize}

\section{Resultados de la fase 1}
\label{subsec:resultados-fase1}

El primer análisis se realizó utilizando una vacante real de Desarrollador Java Junior para la empresa Pragma. El objetivo de esta fase no fue solo medir la precisión, sino estresar el modelo para identificar comportamientos divergentes respecto al juicio humano.

\subsection{Análisis de ranking y puntajes}

A nivel de ordenamiento (ranking), el sistema demostró un desempeño sobresaliente. A pesar de que las magnitudes de los puntajes difirieron, el modelo ordenó a los candidatos exactamente en la misma secuencia que el experto de RR.\,HH., identificando correctamente al mejor candidato (Gabriel Gómez) y discriminando adecuadamente a los perfiles con menor ajuste. La Tabla \ref{tab:ranking-fase1} presenta esta comparación. El detalle completo de los \textit{scores} por criterio y candidato puede consultarse en las Tablas \ref{tab:fase1-eval-parte1}, \ref{tab:fase1-eval-parte2} y \ref{tab:fase1-eval-parte3} del Apéndice de prompts, mientras que la Tabla \ref{tab:fase1-eval-resumen} resume los puntajes finales de modelo y RR.\,HH. para los cinco candidatos evaluados.

\textbf{Nota:} Los resultados completos de todos los análisis, incluyendo puntajes detallados por criterio, razones y comparaciones, están disponibles en formato interactivo en el siguiente \href{https://uniandes-my.sharepoint.com/:x:/r/personal/a_chaparrod_uniandes_edu_co/Documents/Evaluaci\%C3\%B3n\%20del\%20Modelo\%20(3).xlsx?d=w9dadb14c608a4cb699946d70bea075b7&csf=1&web=1&e=cmUJfP}{documento de Excel compartido}.

\begin{table}[H]
    \centering
    \small
    \caption{Comparación de Rankings - Fase 1 (Pragma Java Junior)}
    \label{tab:ranking-fase1}
    \begin{tabular}{|c|l|l|c|}
    \hline
    \textbf{Posición} & \textbf{Ranking Modelo (IA)} & \textbf{Ranking Experto (RR.HH.)} \\ \hline
    1 & Gabriel Gómez & Gabriel Gómez \\ \hline
    2 & Juanchito Bernal & Juanchito Bernal  \\ \hline
    3 & Andrés Chaparro & Andrés Chaparro  \\ \hline
    4 & Juan E. López & Juan E. López \\ \hline
    5 & Javier Barrera & Javier Barrera \\ \hline
    \end{tabular}
\end{table}

\subsubsection*{Hallazgos y discrepancias detectadas}

Posterior al análisis, se realizó una sesión de retroalimentación con el evaluador para analizar por qué, aunque el ranking coincidía, existían diferencias notables en la evaluación de ciertos rubros específicos. Se identificaron cinco hallazgos críticos:

\begin{itemize}
    \item \textbf{Sesgo por ``Datos No Evaluados'' (-1):} El modelo fue excesivamente literal. Si la descripción del trabajo no tenía una sección explícita de Certificaciones, Idiomas o Habilidades Blandas, el sistema marcaba estos rubros como \texttt{-1} (No Evaluado). El experto, en cambio, siempre evaluaba estos aspectos: intuía el idioma por la redacción de la oferta e infería habilidades blandas del contexto. Esto generó una brecha significativa de cobertura en la evaluación.
    
    \item \textbf{Rigurosidad en Educación:} El evaluador humano penalizó fuertemente a los candidatos que no habían culminado sus estudios (no graduados), considerándolo un criterio excluyente para el cargo. El modelo, aunque valoró la educación, fue más laxo al puntuar perfiles en curso.
    
    \item \textbf{Inconsistencia en Responsabilidades:} Se detectó un sesgo en la interpretación de experiencias académicas. Tanto el candidato Juan Esteban López como Andrés Chaparro listaban monitorias académicas; sin embargo, el modelo solo validó positivamente esta experiencia para Juan Esteban, ignorándola para Andrés, lo que evidenció una falta de robustez en la detección de roles equivalentes.
    
    \item \textbf{Fallas en Ubicación:} El comparador de ubicación falló sistemáticamente (0\% de acierto), incapaz de relacionar semánticamente las ubicaciones o de manejar la flexibilidad de trabajo remoto con la misma destreza que el humano.
    
    \item \textbf{Valoración de Certificaciones:} El experto indicó que, aunque una oferta no pida certificaciones explícitamente, la presencia de certificados relevantes en la HV (como AWS o Java) debe sumar puntos positivos como un diferenciador, algo que el modelo ignoraba si el campo de requisitos estaba vacío.
\end{itemize}

\subsection{Ajustes implementados para la Fase 2}

A partir de estos \textit{insights}, se realizaron modificaciones estructurales en el \textit{core} del sistema antes de proceder con los siguientes análisis:

\begin{enumerate}
    \item \textbf{Comparador de certificaciones mejorado:} Se modificó la lógica para que, en ausencia de requisitos de certificación explícitos, el sistema compare las certificaciones del candidato contra las habilidades técnicas requeridas. Este cambio se explica a mayor detalle en \ref{subsec:pipeline-comparacion}.
    \item \textbf{Detección automática de idioma:} Se implementó un paso previo que detecta el idioma de la oferta laboral. Si la oferta está en inglés (aunque no pida Inglés explícitamente), el sistema ahora infiere el requisito. Este cambio se explica a mayor detalle en \ref{subsec:pipeline-jobs}.
    \item \textbf{Inferencia de habilidades blandas:} Se añadió una capa de razonamiento que deduce las habilidades blandas implícitas en las responsabilidades del cargo, evitando que este rubro quede sin evaluar. Este cambio se explica a mayor detalle en \ref{subsec:pipeline-jobs}.
\end{enumerate}

\section{Resultados de la fase 2}
\label{subsec:resultados-fase2}

\subsection{Análisis: Vacante de Practicante en Desarrollo de Software}
\label{subsec:analisis-practicante}

Este escenario planteó un desafío particular: la evaluación de perfiles para una posición de práctica profesional. A diferencia de los roles senior, donde la experiencia laboral es determinante, en estos perfiles el reclutador humano suele priorizar la formación académica en curso, el potencial de aprendizaje y las habilidades blandas, siendo más flexible con la falta de experiencia laboral formal.

\subsubsection*{Comparación de Rankings}

Al contrastar el ordenamiento de candidatos, se observó una divergencia significativa entre el criterio del modelo y el del experto de RR.HH., llegando a una inversión casi total en los extremos del ranking, como se evidencia en la Tabla \ref{tab:ranking-fase2-practicante}.

\begin{table}[H]
    \centering
    \small
    \caption{Comparación de Rankings - Análisis 2.1 (Practicante)}
    \label{tab:ranking-fase2-practicante}
    \begin{tabular}{|c|l|l|}
    \hline
    \textbf{Posición} & \textbf{Ranking Modelo (IA)} & \textbf{Ranking Experto (RR.HH.)} \\ \hline
    1 & María Ramírez & Andrés Chaparro \\ \hline
    2 & Stiven Viedman & Carlos Santoyo  \\ \hline
    3 & Andrés Chaparro & Stiven Viedman  \\ \hline
    4 & William Xavier & William Xavier \\ \hline
    5 & Carlos Santoyo & María Ramírez \\ \hline
    \end{tabular}
\end{table}

Mientras que para el experto humano los candidatos más fuertes fueron Andrés Chaparro y Carlos Santoyo (debido a su perfil educativo), el modelo priorizó a María Ramírez, quien fue la candidata mas descartada por el humano (posición 5).

\subsubsection*{Análisis de puntajes y discrepancias}

A partir de la revisión cualitativa con el evaluador, se identificaron las causas raíz de estas discrepancias:

\begin{itemize}
    \item \textbf{Sesgo de Normalización (Caso María Ramírez):} La candidata María Ramírez presentaba una hoja de vida con muy poca información, lo que resultó en que muchos aspectos (Experiencia, Responsabilidades, Certificaciones) fueran marcados como \texttt{-1} (No evaluado). Debido a la lógica de normalización del sistema (explicada en la Sección \ref{chap:metodologia}), su puntaje final se calculó únicamente sobre los pocos aspectos presentes (principalmente Habilidades Técnicas e Idiomas), donde tuvo coincidencias altas. Matemáticamente obtuvo un 74\%, pero para el reclutador humano, la ausencia de información es un criterio de descarte (33\%), evidenciando una limitación del algoritmo al manejar perfiles incompletos.
    
    \item \textbf{Criterio de Experiencia en Practicantes:} El experto humano valoró positivamente a candidatos como Andrés y Carlos por tener experiencia laboral previa y estar cursando carreras afines. El modelo, aunque puntuó bien la experiencia y habilidades técnicas de Andrés (95\% y 100\% respectivamente), lo penalizó severamente en el rubro de Educación (30\% vs 90\% del humano) y Ubicación (0\%). El modelo no logró interpretar con la misma flexibilidad que el humano que, para un practicante, estar "en curso" de una ingeniería es el estado ideal, y aplicó criterios de titulación más rígidos.
    
    \item \textbf{Persistencia del error de Ubicación:} A pesar de los ajustes, el comparador de ubicación continuó fallando (0\% en la mayoría de casos) al no lograr correlacionar ciudades implícitas o regiones cercanas con la misma destreza que el humano, o debido a la falta de especificación explícita en los CVs.
\end{itemize}

\subsection{Análisis: Vacante Sezzle A.I. Engineering Intern}
\label{subsec:analisis-sezzle-intern}

El segundo análisis de la Fase 2 evaluó el desempeño del sistema en una vacante internacional para un rol de pasantía en Inteligencia Artificial. Al igual que en el caso anterior, este rol privilegia el potencial y la formación académica por encima de la experiencia laboral extensa, pero con un componente técnico mucho más exigente en términos de herramientas modernas (Python, SQL, Cloud).

\subsubsection*{Comparación de Rankings}

En este caso, se observó una alineación más sólida en la parte superior de la tabla entre el modelo y el experto humano. Ambos sistemas de evaluación identificaron a Julián Galindo como el candidato más fuerte, aunque hubo discrepancias en el ordenamiento de los candidatos intermedios.

\begin{table}[H]
    \centering
    \small
    \caption{Comparación de Rankings - Análisis 2.2 (Sezzle AI Intern)}
    \label{tab:ranking-fase2-sezzle-intern}
    \begin{tabular}{|c|l|l|}
    \hline
    \textbf{Posición} & \textbf{Ranking Modelo (IA)} & \textbf{Ranking Experto (RR.HH.)} \\ \hline
    1 & Julián Galindo & Julián Galindo \\ \hline
    2 & Andrés Chaparro & Juan Esteban López \\ \hline
    3 & Juan Esteban López & Andrés Chaparro \\ \hline
    4 & Javier Barrera & Carlos Santoyo \\ \hline
    5 & Carlos Santoyo & Javier Barrera \\ \hline
    \end{tabular}
\end{table}

\subsubsection*{Análisis de puntajes y discrepancias}

A partir de la revisión cualitativa con el evaluador, se identificaron las causas raíz de estas discrepancias:

\begin{itemize}
    \item \textbf{Mejora en Skills Técnicos y Blandos:} El evaluador destacó que el modelo realizó un análisis muy preciso de las habilidades técnicas y blandas, así como de las certificaciones. Las razones generadas por la IA fueron coherentes con el juicio experto, identificando correctamente las brechas en tecnologías específicas.
    
    \item \textbf{Confusión por Modalidad Remota:} El comparador de ubicación falló consistentemente (0\% en los 5 casos). El error se atribuye a que la vacante especificaba la ubicación como ``Colombia, Remote``. El modelo, al no encontrar una coincidencia exacta de ciudad en las hojas de vida (que listan ciudades específicas como Bogotá), no logró resolver la relación de contención geográfica implícita.
    
    \item \textbf{Penalización de Experiencia en Pasantías:} El experto humano indicó que, para esta vacante de practicante, la experiencia laboral no era un factor crítico, priorizando la educación en curso. El modelo, sin embargo, asignó puntajes medios (alrededor del 50\%) en experiencia, señalando la falta de roles previos similares. Aunque técnicamente correcto, esto redujo el promedio general de los candidatos fuertes frente al criterio humano, que fue más indulgente en este rubro.
\end{itemize}


\subsection{Análisis: Vacante Sezzle Junior Software Engineer}
\label{subsec:analisis-sezzle-junior}

El tercer y último análisis se centró en un rol Junior de Ingeniería de Software. A diferencia de las pasantías, este perfil sí requiere experiencia profesional (típicamente 1-3 años) y título profesional, lo que permitió evaluar cómo el sistema maneja criterios de descarte más estrictos.

\subsection*{Comparación de Rankings}

En este escenario, el sistema alcanzó su mayor nivel de precisión en el ranking, coincidiendo perfectamente con el experto humano en la identificación del mejor candidato (Juanchito Bernal) y manteniendo una coherencia notable en el resto de la tabla, salvo por una inversión en las posiciones intermedias.

\begin{table}[H]
    \centering
    \small
    \caption{Comparación de Rankings - Análisis 2.3 (Sezzle Junior Dev)}
    \label{tab:ranking-fase2-sezzle-junior}
    \begin{tabular}{|c|l|l|}
    \hline
    \textbf{Posición} & \textbf{Ranking Modelo (IA)} & \textbf{Ranking Experto (RR.HH.)} \\ \hline
    1 & Juanchito Bernal & Juanchito Bernal \\ \hline
    2 & Andrés Chaparro & Gabriel Gómez \\ \hline
    3 & Gabriel Gómez & Stiven Viedman \\ \hline
    4 & Stiven Viedman & Andrés Chaparro \\ \hline
    5 & Juan Esteban López & Juan Esteban López \\ \hline
    \end{tabular}
\end{table}


A partir de la revisión cualitativa con el evaluador, se identificaron las causas raíz de estas discrepancias:


\begin{itemize}
    \item \textbf{Penalización por "No Graduado":} La mayor divergencia se dio con el candidato Andrés Chaparro (+28\%). Mientras el modelo valoró sus habilidades (57\%), el experto humano lo castigó severamente (29\%) por no haber culminado sus estudios, considerándolo un criterio de descarte (knock-out) para una posición Junior full-time, a diferencia de una pasantía.
    
    \item \textbf{Fenómeno de Sobrecalificación:} El candidato Gabriel Gómez, quien posee un perfil senior, fue puntuado por el humano con un 64\%, dejándolo en segundo lugar. El reclutador indicó que, aunque técnicamente apto, su exceso de experiencia podría implicar expectativas salariales altas, haciéndolo "sobrecalificado". El modelo, al no tener contexto salarial, simplemente evaluó el ajuste técnico, otorgándole un puntaje competitivo (51\%) aunque inferior al candidato ideal, mostrando una alineación interesante con la intuición humana por razones distintas.
    
    \item \textbf{Precisión en Idiomas:} El desempeño del comparador de idiomas fue validado positivamente. El experto notó que la calificación del modelo variaba acordemente al nivel detectado, alineándose con su propio criterio.
    
    \item \textbf{Persistencia del error de Ubicación:} Al igual que en el caso anterior, la ubicación "Remote" generó falsos negativos en todos los candidatos, confirmando la necesidad de mejorar la lógica de inferencia geográfica para modalidades de teletrabajo.
\end{itemize}

\section{Discusión general de resultados}

La ejecución de las dos fases experimentales permite concluir que el sistema propuesto logra un nivel de competencia significativo como herramienta de apoyo a la decisión. En términos de ranking, el modelo demostró una capacidad robusta para identificar a los candidatos más fuertes, coincidiendo con el experto humano en el primer lugar en 2 de los 3 análisis de la Fase 2.

Sin embargo, las discrepancias en los puntajes absolutos revelan que el juicio humano incorpora reglas de negocio tácitas (como la penalización severa por no estar graduado en roles Junior o la devaluación de la experiencia en roles de pasante) que el modelo, en su configuración actual, no pondera con la misma agresividad. Mientras el sistema evalúa la compatibilidad semántica pura, el reclutador evalúa la viabilidad de contratación.

A pesar de estas diferencias, la validación experta confirmó que los desgloses explicativos (razones) generados por la IA son coherentes y útiles, ofreciendo una transparencia que facilita al reclutador entender por qué un candidato obtuvo cierto puntaje, permitiéndole ajustar la decisión final con su propio criterio estratégico.
}{\chapter{Experimentos y Resultados} Contenido placeholder.}
\IfFileExists{chapters/07-discusion-amenazas.tex}{\chapter{Discusión y Amenazas a la Validez}
\label{chap:discusion}

\section{Interpretación de Resultados}
Implicaciones prácticas y teóricas.

\section{Limitaciones y Amenazas}
Interna, externa, de constructo y de conclusión estadística.

\section{Consideraciones Éticas}
Sesgos, transparencia y explicabilidad.


}{\chapter{Discusión y Amenazas a la Validez} Contenido placeholder.}
\IfFileExists{chapters/08-conclusiones-trabajo-futuro.tex}{\chapter{Conclusiones y Trabajo Futuro}
\label{chap:conclusiones}

\section{Conclusiones}

El desarrollo y evaluación de este prototipo ha permitido validar la viabilidad técnica de utilizar Grandes Modelos de Lenguaje (LLMs) como motor central de un sistema de recomendación para reclutamiento. A partir de la experimentación, se derivan las siguientes conclusiones principales:

\begin{enumerate}
    \item \textbf{Eficacia en la Estructuración de Datos:} Si bien el sistema demostró una alta competencia en la extracción de información con una precisión superior al 90\%, también exhibió tendencias a confundirse en perfiles con información estructurada de forma no estandar al combinar diferentes partes de la hoja de vida como proyectos de portafolio con experiencia.

    \item \textbf{Criterio humano} La herramienta demostró ser un filtro eficaz, coincidiendo frecuentemente con el experto en la identificación de los mejores perfiles y superándolo ocasionalmente en la detección de detalles técnicos específicos. Sin embargo, el modelo carece de la capacidad de ``pensar el problema'' con la visión que tiene el reclutador. Su eficacia es semántica, pero falla al interpretar las dinámicas de negocio como la flexibilidad en la experiencia para practicas o la falta de titulos en vacantes junior.


    \item \textbf{La Brecha del Contexto de Negocio:} Se identificó una limitación crítica en la asignación de puntajes absolutos. Mientras el modelo evalúa cada criterio por separado con su peso asignado, los procesos reales de selección humana operan sin pesos costantes para cada criterio (ej. "si no tiene título, no sirve"). Esta discrepancia generó diferencias en los puntajes finales, evidenciando que el modelo necesita ser acotada por reglas de negocio.

    \item \textbf{Valor de la Explicabilidad:} Uno de los mayores aportes percibido por el usuario RR.HH. fue la capacidad del sistema para generar justificaciones textuales. Esta característica transforma la herramienta de una ``caja negra'' a un asistente transparente, permitiendo al reclutador validar rápidamente por qué un candidato fue recomendado y descubriendo coincidencias técnicas que podrían haber pasado desapercibidas en una lectura manual rápida.

    \item \textbf{Eficiencia Operativa:} La implementación del sistema no solo reduce de manera significativa el tiempo de análisis, sino que también redistribuye el esfuerzo del reclutador. En lugar de dedicar tiempo a extraer información manualmente, el analista pasa a validar y ajustar decisiones, fortaleciendo el componente estratégico del proceso de selección.

\end{enumerate}

\section{Trabajo Futuro}

Basado en las limitaciones identificadas y las oportunidades de mejora, se proponen las siguientes líneas de investigación y desarrollo:

\subsection{Inyección de Reglas de Negocio mediante RAG (Retrieval-Augmented Generation)}
La principal carencia detectada fue la falta de conocimiento sobre las reglas específicas de cada empresa (políticas de trabajo remoto, requisitos de visado, obligatoriedad de títulos, bandas salariales). Para solucionar esto, se propone implementar una arquitectura RAG.
\begin{itemize}
    \item \textbf{Propuesta:} Integrar una base de conocimiento vectorial que contenga los manuales de contratación y políticas internas de la empresa. Antes de realizar el análisis de un candidato, el sistema recuperaría las reglas de negocio relevantes para esa vacante específica y las inyectaría en el contexto del modelo.v Esto permitiría que el modelo penalice o descarte candidatos no solo por falta de habilidades técnicas, sino por incumplimiento de normativas corporativas.
\end{itemize}

\subsection{Aprendizaje Activo (Human-in-the-Loop Feedback)}
Actualmente, el sistema es estático: si el reclutador no está de acuerdo con una recomendación, el modelo no aprende de ese error.
\begin{itemize}
    \item \textbf{Propuesta:} Implementar un mecanismo de retroalimentación donde las correcciones del reclutador (ej. cambiar un score manualmente o reordenar el ranking) se almacenen y utilicen para afinar futuros análisis. Esto implicaria que el modelo tras diversas iteraciones se puede adaptar al contexto y mejorar sus resultados.
\end{itemize}

\subsection{Análisis Multimodal de Candidatos}
El alcance actual se limita al texto. Sin embargo, los procesos de selección modernos incluyen portafolios visuales (GitHub, Behance) y video-entrevistas.
\begin{itemize}
    \item \textbf{Propuesta:} Extender la capacidad de extracción para analizar enlaces externos. Por ejemplo, utilizar agentes que naveguen a los repositorios de código del candidato para evaluar la calidad real de sus proyectos, o transcribir y analizar video-presentaciones para inferir habilidades de comunicación con mayor precisión.
\end{itemize}

\subsection{Validación en Dominios No Tecnológicos}
El estudio se limitó al nicho de desarrollo de software. Sería valioso replicar el experimento en sectores con habilidades más subjetivas o "blandas", como ventas, diseño gráfico o psicología, para evaluar si la capacidad de inferencia semántica del modelo se mantiene efectiva cuando las competencias no son tan estandarizadas.
}{\chapter{Conclusiones y Trabajo Futuro} Contenido placeholder.}
\makeatother

% ======= Apéndices =======
\appendix
\makeatletter
\IfFileExists{appendices/A-figuras-aplicacion.tex}{\usepackage{float}

\chapter{Figuras de la aplicación}
\begin{figure}[H]
    \centering
    \includegraphics[width=0.5\linewidth]{Secuencia_Subir_Cv.png}
    \caption{Secuencia: Subir CV (POST /cvs)}
    \label{fig:seq-subir-cv}
\end{figure}

\begin{figure}[H]
    \centering
    \includegraphics[width=0.5\linewidth]{Secuencia_Crear_Job.png}
    \caption{Secuencia: Crear Job (POST /jobs)}
    \label{fig:seq-crear-job}
\end{figure}

\begin{figure}[H]
    \centering
    \includegraphics[width=0.5\linewidth]{SecuenciaAnalizar.png}
    \caption{Secuencia: Analizar CV vs Job (POST /analyze/\{cv\_id\}/\{job\_id\})}
    \label{fig:seq-analizar}
\end{figure}

\begin{figure}
    \centering
    \includegraphics[width=0.5\linewidth]{Flujo_de_Persistencia.png}
    \caption{Flujo de Persistencia (Ejemplo con Analysis)}
    \label{fig:flujo_persistencia_analysis}
\end{figure}
\begin{figure}[H]
    \centering
    \includegraphics[width=0.5\linewidth]{Actividad_Protocolo_de_evaluacion.png}
    \caption{Actividad: Protocolo de evaluación}
    \label{fig:actividad-evaluacion}
\end{figure}

\begin{figure}[H]
    \centering
    \includegraphics[width=0.5\linewidth]{Mapa_de_endpoints.png}
    \caption{Mapa de endpoints}
    \label{fig:mapa-endpoints}
\end{figure}
\begin{figure}[H]
    \centering
    \includegraphics[width=0.5\linewidth]{DetallesImplementacionDatos.png}
    \caption{Detalles de la implementación en la capa de datos}
    \label{fig:DetallesImplementacionDatos}
\end{figure}

\begin{figure}[H]
    \centering
    \includegraphics[width=0.5\linewidth]{Flujo_Inferir_Habilidades_Blandas.png}
    \caption{Flujo inferir habilidades blandas en descripciones}
    \label{fig:Flujo_Inferir_Habilidades_Blandas}
\end{figure}

\begin{figure}[H]
    \centering
    \includegraphics[width=0.5\linewidth]{Flujo_Inferir_Lenguaje.png}
    \caption{Flujo inferir lenguaje en descripciones}
    \label{fig:Flujo_Inferir_Lenguaje}
\end{figure}

\begin{figure}[H]
    \centering
    \includegraphics[width=0.5\linewidth]{Flujo_Comparaciones_Certificaciones.png}
    \caption{Flujo de comparaciones de certificaciones CV vs Descripción}
    \label{fig:flujo-comparacion-certificaciones}
\end{figure}

\begin{figure}[H]
    \centering
    \includegraphics[width=0.5\linewidth]{SubirOfertaDeTrabajo.png}
    \caption{Interfaz: Vista subir oferta de trabajo}
    \label{fig:ui-crear-job}
\end{figure}
\begin{figure}[H]
    \centering
    \includegraphics[width=0.5\linewidth]{Vista_Subir_CV.png}
    \caption{Interfaz: Vista subir Hoja de Vida}
    \label{fig:ui-subir-cv}
\end{figure}
\begin{figure}[H]
    \centering
    \includegraphics[width=0.5\linewidth]{AnalizarCv_vs_Job.png}
    \caption{Interfaz: Vista analizar CV vs Job}
    \label{fig:ui-analizar}
\end{figure}
\begin{figure}[H]
    \centering
    \includegraphics[width=0.5\linewidth]{Vista_Listar_CVs.png}
    \caption{Interfaz: Vista Listar Hojas de Vida}
    \label{fig:ui-listar-cvs}
\end{figure}
\begin{figure}[H]
    \centering
    \includegraphics[width=0.5\linewidth]{Vista_Listar_Descripciones.png}
    \caption{Interfaz: Vista Listar Descripciones}
    \label{fig:ui-listar-jobs}
\end{figure}

\begin{figure}[H]
    \centering
    \includegraphics[width=0.5\linewidth]{Vista_Historial_Analisis.png}
    \caption{Interfaz: Vista Historial Analisis}
    \label{fig:ui-historial-analisis}
\end{figure}

\begin{figure}[H]
    \centering
    \includegraphics[width=0.5\linewidth]{TopCandidatos.png}
    \caption{Interfaz: Vista Ranking de candidatos}
    \label{fig:ui-ranking}
\end{figure}

\begin{figure}[H]
    \centering
    \includegraphics[width=0.5\linewidth]{VistaResumenResultadoAnalisis.png}
    \caption{Interfaz: Vista Resumen Resultado Análisis}
    \label{fig:ui-resumen-analisis}
\end{figure}
\begin{figure}[H]
    \centering
    \includegraphics[width=0.50\linewidth]{Vista_Resultados_Comparacion.png}
    \caption{Interfaz: Vista de resultados y desglose por aspecto}
    \label{fig:frontend-resultados}
\end{figure}

\begin{figure}[H]
    \centering
    \includegraphics[width=0.5\linewidth]{VistaGraficaAnalisis.png}
    \caption{Interfaz: Vista Grafica Score Obtenido vs Porcentaje de Criterios Evaluados}
    \label{fig:ui-grafica-score}
\end{figure}

\begin{figure}[H]
    \centering
    \includegraphics[width=0.5\linewidth]{Vista_CV_Detalle_InfoPersonal.png}
    \caption{Interfaz: Vista del detalle de información personal de la Hoja de Vida}
    \label{fig:ui-cv-info-personal}
\end{figure}

\begin{figure}[H]
    \centering
    \includegraphics[width=0.5\linewidth]{Vista_CV_Detalle_Educacion.png}
    \caption{Interfaz: Vista del detalle de educación de la Hoja de Vida}
    \label{fig:ui-cv-educacion}
\end{figure}

\begin{figure}[H]
    \centering
    \includegraphics[width=0.5\linewidth]{Vista_CV_Detalle_Experiencia.png}
    \caption{Interfaz: Vista del detalle de experiencia laboral de la Hoja de Vida}
    \label{fig:ui-cv-experiencia}
\end{figure}

\begin{figure}[H]
    \centering
    \includegraphics[width=0.5\linewidth]{Vista_CV_Detalles_Habilidades.png}
    \caption{Interfaz: Vista del detalle de habilidades técnicas de la Hoja de Vida}
    \label{fig:ui-cv-habilidades}
\end{figure}

\begin{figure}[H]
    \centering
    \includegraphics[width=0.5\linewidth]{Vista_CV_Detalles_Certificaciones.png}
    \caption{Interfaz: Vista del detalle de certificaciones de la Hoja de Vida}
    \label{fig:ui-cv-certificaciones}
\end{figure}

\begin{figure}[H]
    \centering
    \includegraphics[width=0.5\linewidth]{Vista_CV_Detalles_Idiomas.png}
    \caption{Interfaz: Vista del detalle de idiomas de la Hoja de Vida}
    \label{fig:ui-cv-idiomas}
\end{figure}

\begin{figure}[H]
    \centering
    \includegraphics[width=0.5\linewidth]{Vista_CV_Detalle_AnalisisHechosPorCV.png}
    \caption{Interfaz: Vista del detalle de análisis realizados para una Hoja de Vida}
    \label{fig:ui-cv-analisis-realizados}
\end{figure}

\begin{figure}[H]
    \centering
    \includegraphics[width=0.5\linewidth]{Vista_Descripcion_Detalle_InformacionGeneral.png}
    \caption{Interfaz: Vista la información general de una descripción}
    \label{fig:ui-job-informacion-general}
\end{figure}

\begin{figure}[H]
    \centering
    \includegraphics[width=0.5\linewidth]{Vista_Descripcion_Detalles_Requisitos.png}
    \caption{Interfaz: Vista de los requisitos de una descripción}
    \label{fig:placeholder}
\end{figure}
}{\chapter{Figuras de la aplicación} Contenido placeholder.}
\IfFileExists{appendices/B-prompts.tex}{\chapter{Prompts utilizados}

Este apéndice documenta los prompts completos utilizados en el sistema para la estructuración de hojas de vida, descripciones de trabajo y evaluación de compatibilidad. Todos los prompts emplean una arquitectura de dos mensajes: un \textbf{System Message} (que define el rol del modelo) y un \textbf{Human Message} (que contiene la instrucción específica y el texto a procesar).


\section*{Estructuración de CVs}

\subsection*{System Message (Rol común para todos los prompts de CV)}
\begin{tcolorbox}[colback=gray!5!white, colframe=gray!75!black, title=System Message]
\texttt{Eres un experto en extraer información estructurada de hojas de vida. IMPORTANTE: Responde ÚNICAMENTE con la estructura que se te pide, sin texto adicional, sin explicaciones, sin bloques de código markdown (```json).}
\end{tcolorbox}

\textit{Nota:} Cada prompt se complementa con \texttt{Texto del CV: \{text\}}, donde \texttt{\{text\}} es el texto limpio extraído del PDF.


\subsection*{Prompts específicos por campo}

\begin{table}[H]
  \centering
  \caption{Prompts completos de estructuración de CV}
  \label{tab:prompts-cv}
  \scriptsize
  \begin{tabular}{@{}lp{10cm}@{}}
    \toprule
    \textbf{Campo} & \textbf{Prompt (Human Message)} \\
    \midrule
    
    \textbf{Personal} & 
    Extrae únicamente la información personal básica del siguiente CV. Responde en formato JSON con esta estructura exacta: \texttt{\{``name'': ``nombre completo'', ``email'': ``correo electrónico'', ``phone'': ``número de teléfono'', ``location'': ``ubicación/ciudad''\}}. Si no encuentras alguna información, deja el campo como DESCONOCIDO (\texttt{``DESCONOCIDO''}). \\
    \addlinespace
    
    \textbf{Educación} & 
    Extrae únicamente la información educativa del siguiente CV. Responde en formato JSON como un array de objetos con esta estructura: \texttt{[ \{``degree'': ``título obtenido'', ``institution'': ``nombre de la institución'', ``year'': ``año de graduación'', ``field'': ``campo de estudio''\} ]}. Si no hay información educativa, devuelve un array vacío \texttt{[]}. \\
    \addlinespace
    
    \textbf{Experiencia} & 
    Extrae únicamente la experiencia laboral del siguiente CV. Responde en formato JSON como un array de objetos con esta estructura: \texttt{[ \{``position'': ``cargo o puesto'', ``company'': ``nombre de la empresa'', ``duration'': ``duración del trabajo'', ``description'': ``descripción breve de responsabilidades''\} ]}. Si no hay experiencia laboral, devuelve un array vacío \texttt{[]}. \\
    \addlinespace
    
    \textbf{Habilidades técnicas} & 
    Extrae únicamente las habilidades técnicas (technical\_skills) del siguiente CV. Responde en formato array \texttt{[]} de strings: \texttt{[``habilidad1'', ``habilidad2'', ``habilidad3'']}. Si no hay habilidades técnicas, devuelve un array vacío \texttt{[]}. \\
    \addlinespace
    
    \textbf{Habilidades blandas} & 
    Extrae únicamente las habilidades blandas (soft\_skills) del siguiente CV. Responde en formato array \texttt{[]} de strings: \texttt{[``habilidad1'', ``habilidad2'', ``habilidad3'']}. Si no hay habilidades blandas, devuelve un array vacío \texttt{[]}. \\
    \addlinespace
    
    \textbf{Certificaciones} & 
    Extrae únicamente las certificaciones del siguiente CV. Responde en formato JSON como un array de objetos con esta estructura: \texttt{[ \{``name'': ``nombre de la certificación'', ``issuer'': ``institución que la emitió'', ``year'': ``año de obtención''\} ]}. Si no hay certificaciones, devuelve un array vacío \texttt{[]}. \\
    \addlinespace
    
    \textbf{Idiomas} & 
    Extrae únicamente los idiomas del siguiente texto. Responde en formato JSON como un objeto donde las llaves son los idiomas y los valores son los niveles: \texttt{\{``idioma1'': ``nivel1'', ``idioma2'': ``nivel2''\}}. Si no hay idiomas, devuelve un objeto vacío \texttt{\{\}}. \\
    
    \bottomrule
  \end{tabular}
\end{table}



\section*{Estructuración de Jobs}

\subsection*{System Message (Rol común para todos los prompts de Job)}
\begin{tcolorbox}[colback=gray!5!white, colframe=gray!75!black, title=System Message]
\texttt{Eres un experto en extraer información estructurada de descripciones de trabajo. IMPORTANTE: Responde ÚNICAMENTE con la estructura que se te pide, sin texto adicional, sin explicaciones, sin bloques de código markdown (```json).}
\end{tcolorbox}
\textit{Nota:} Cada prompt se complementa con \texttt{Texto del CV: \{text\}}, donde \texttt{\{text\}} es el texto limpio extraído de la descripción de trabajo.
\subsection*{Prompts específicos por campo}

\begin{table}[H]
  \centering
  \caption{Prompts completos de estructuración de descripciones de trabajo}
  \label{tab:prompts-job}
  \scriptsize
  \begin{tabular}{@{}lp{10cm}@{}}
    \toprule
    \textbf{Campo} & \textbf{Prompt (Human Message)} \\
    \midrule
    
    \textbf{Información básica} & 
    Extrae únicamente la información básica del trabajo del siguiente texto. Responde en formato JSON con esta estructura exacta: \texttt{\{``job\_title'': ``título del puesto'', ``company\_name'': ``nombre de la empresa'', ``work\_modality'': ``modalidad de trabajo (presencial, remoto, híbrido)'', ``contract\_type'': ``tipo de contrato (tiempo completo, medio tiempo, etc.)'', ``salary'': ``información salarial'', ``summary'': ``resumen breve del puesto''\}}. Si no encuentras alguna información, deja el campo como DESCONOCIDO (\texttt{``DESCONOCIDO''}). \\
    \addlinespace
    
    \textbf{Responsabilidades} & 
    Extrae únicamente las responsabilidades y funciones del cargo del siguiente texto. Responde en formato array \texttt{[]} de strings: \texttt{[``responsabilidad1'', ``responsabilidad2'', ``responsabilidad3'']}. Si no hay responsabilidades, devuelve un array vacío \texttt{[]}. \\
    \addlinespace
    
    \textbf{Ubicación} & 
    Extrae únicamente la ubicación del trabajo del siguiente texto. Responde en formato JSON con esta estructura: \texttt{\{``location'': ``ubicación del trabajo''\}}. Si no encuentras la ubicación, devuelve \texttt{\{``location'': ``DESCONOCIDO''\}}. \\
    \addlinespace
    
    \textbf{Educación} & 
    Extrae únicamente los requisitos educativos del siguiente texto. Responde en formato JSON con esta estructura: \texttt{\{``education'': ``requisitos educativos''\}}. Si no encuentras requisitos educativos, devuelve \texttt{\{``education'': ``DESCONOCIDO''\}}. \\
    \addlinespace
    
    \textbf{Experiencia} & 
    Extrae únicamente los requisitos de experiencia del siguiente texto. Responde en formato JSON con esta estructura: \texttt{\{``experience'': ``requisitos de experiencia''\}}. Si no encuentras requisitos de experiencia, devuelve \texttt{\{``experience'': ``DESCONOCIDO''\}}. \\
    \addlinespace
    
    \textbf{Habilidades técnicas} & 
    Extrae únicamente las habilidades técnicas requeridas del siguiente texto. Responde en formato array \texttt{[]} de strings: \texttt{[``habilidad1'', ``habilidad2'', ``habilidad3'']}. Si no hay habilidades técnicas, devuelve un array vacío \texttt{[]}. \\
    \addlinespace
    
    \textbf{Habilidades blandas} & 
    Extrae únicamente las habilidades blandas requeridas del siguiente texto. Responde en formato array \texttt{[]} de strings: \texttt{[``habilidad1'', ``habilidad2'', ``habilidad3'']}. Si no hay habilidades blandas, devuelve un array vacío \texttt{[]}. \\
    \addlinespace
    
    \textbf{Certificaciones} & 
    Extrae únicamente las certificaciones requeridas del siguiente texto. Responde en formato array \texttt{[]} de strings: \texttt{[``certificado1'', ``certificado2'', ``certificado3'']}. Si no encuentras certificaciones, devuelve un array vacío \texttt{[]}. \\
    \addlinespace
    
    \textbf{Idiomas} & 
    Extrae únicamente los requisitos de idiomas del siguiente texto. Responde en formato JSON como un objeto donde las llaves son los idiomas y los valores son los niveles: \texttt{\{``idioma1'': ``nivel1'', ``idioma2'': ``nivel2''\}}. Si no hay requisitos de idiomas, devuelve un objeto vacío \texttt{\{\}}. \\
    \addlinespace
    
    \textbf{Beneficios} & 
    Extrae únicamente los beneficios ofrecidos del siguiente texto. Responde en formato array \texttt{[]} de strings: \texttt{[``beneficio1'', ``beneficio2'', ``beneficio3'']}. Si no hay beneficios, devuelve un array vacío \texttt{[]}. \\
    
    \bottomrule
  \end{tabular}
\end{table}



\subsection*{Prompts de inferencia para descripciones }

Estos prompts se activan automáticamente solo cuando la extracción directa no encuentra información.

\begin{table}[H]
  \centering
  \caption{Prompts de fallback para inferencia}
  \label{tab:prompts-job-fallback}
  \scriptsize
  \begin{tabular}{@{}lp{10cm}@{}}
    \toprule
    \textbf{Campo} & \textbf{Prompt (Human Message)} \\
    \midrule
    
    \textbf{Inferencia soft skills} & 
    \textbf{Contexto:} Se activa solo si la extracción directa de soft skills devuelve una lista vacía. \par
    \textbf{Prompt:} Eres un experto en recursos humanos. Analiza las responsabilidades de este trabajo e infiere SOLO las soft skills MÁS CRÍTICAS y NECESARIAS. \par
    Título del trabajo: \texttt{\{job\_title\}} \par
    Responsabilidades: \texttt{- \{responsabilidad1\}, - \{responsabilidad2\}, ...} \par
    IMPORTANTE: Infiere MÁXIMO 3-5 soft skills (no más), si es necesario no pongas ninguna. Solo las MÁS IMPORTANTES y EVIDENTES. NO inventes soft skills genéricas. Basate únicamente en lo que las responsabilidades requieren claramente. \par
    Responde ÚNICAMENTE con un array JSON de strings (nombres de soft skills en español): \texttt{[``soft skill 1'', ``soft skill 2'', ``soft skill 3'']}. Sin bloques de código markdown. \par
     \\
    \addlinespace
    
    \textbf{Detección de idioma} & 
    \textbf{Contexto:} Se activa solo si la extracción directa de idiomas devuelve un objeto vacío. \par
    \textbf{Prompt:} Detecta el idioma principal en el que está escrito el siguiente texto. Responde ÚNICAMENTE con un objeto JSON donde la clave es el nombre del idioma en español y el valor es ``Intermedio'' (como nivel básico requerido). \par
    Formato: \texttt{\{``idioma'': ``Intermedio''\}} \par
    Texto: \texttt{\{primeros 1000 caracteres del texto\}} \par
    Responde solo con el JSON, sin bloques de código markdown. \par
    \\
    
    \bottomrule
  \end{tabular}
\end{table}


\section*{Comparadores}

Los comparadores utilizan prompts de Role Prompting para evaluar la compatibilidad entre CVs y ofertas laborales en distintas dimensiones. A diferencia de los extractores, cada comparador define su propio System Message específico según el aspecto a evaluar.

\textit{Nota:} Todos los comparadores reciben como entrada el CV completo estructurado y la descripción de trabajo completa estructurada en formato JSON.

\subsection*{Prompts de comparadores por aspecto}

\begin{table}[H]
  \centering
  \caption{Prompts completos de comparadores por aspecto}
  \label{tab:prompts-comp}
  \scriptsize
  \begin{tabular}{@{}lp{10cm}@{}}
    \toprule
    \textbf{Aspecto} & \textbf{System Message y Prompt} \\
    \midrule
    
    \textbf{Habilidades técnicas} & 
    \textbf{System Message:} Eres un experto en evaluar compatibilidad de habilidades técnicas entre candidatos y ofertas de trabajo. \par
    \textbf{Prompt:} Compara las habilidades técnicas del CV con las requeridas en la oferta. Considera coincidencias exactas, tecnologías similares y habilidades transferibles. Responde solo con JSON: \texttt{\{``score'': 0--1, ``reason'': ``justificación''\}}. \\
    \addlinespace
    
    \textbf{Experiencia} & 
    \textbf{System Message:} Eres un experto en evaluar experiencia laboral relevante. \par
    \textbf{Prompt:} Evalúa la compatibilidad entre la experiencia del candidato y los requisitos del puesto, considerando años de experiencia, roles previos similares y tecnologías utilizadas. Responde solo con JSON: \texttt{\{``score'': 0--1, ``reason'': ``justificación''\}}. \\
    \addlinespace
    
    \textbf{Responsabilidades} & 
    \textbf{System Message:} Eres un experto en analizar compatibilidad entre responsabilidades de cargo y experiencia profesional. \par
    \textbf{Prompt:} Compara las responsabilidades del puesto con la experiencia del candidato. Evalúa si las funciones previas del candidato se alinean con las responsabilidades requeridas. Responde solo con JSON: \texttt{\{``score'': 0--1, ``reason'': ``justificación''\}}. \\
    \addlinespace
    
    \textbf{Educación} & 
    \textbf{System Message:} Eres un experto en evaluar compatibilidad educativa entre perfiles y requisitos. \par
    \textbf{Prompt:} Evalúa si la formación académica del candidato cumple con los requisitos educativos del puesto. Considera nivel de titulación, campo de estudio y relevancia. Responde solo con JSON: \texttt{\{``score'': 0--1, ``reason'': ``justificación''\}}. \\
    \addlinespace
    
    \textbf{Idiomas} & 
    \textbf{System Message:} Eres un experto en evaluar niveles de idioma y compatibilidad lingüística. \par
    \textbf{Prompt:} Compara los idiomas y niveles de dominio del candidato con los requeridos por el puesto. Evalúa si el nivel del candidato cumple o supera el nivel requerido para cada idioma. Responde solo con JSON: \texttt{\{``compatible'': true/false, ``score'': 0--1, ``reason'': ``justificación''\}}. \\
    \addlinespace
    
    \textbf{Ubicación} & 
    \textbf{System Message:} Eres un experto en evaluar compatibilidad geográfica. \par
    \textbf{Prompt:} Evalúa la proximidad geográfica entre la ubicación del candidato y la ubicación del trabajo. Considera: misma ciudad = 1.0, mismo país = 0.7, países cercanos = 0.3--0.5, países lejanos = 0.0--0.3. Responde solo con JSON: \texttt{\{``score'': 0--1, ``reason'': ``justificación''\}}. \\
    \addlinespace
    
    \textbf{Habilidades blandas} & 
    \textbf{System Message:} Eres un experto en evaluar habilidades blandas y competencias interpersonales. \par
    \textbf{Prompt:} Compara las habilidades blandas del candidato con las requeridas en el puesto. Evalúa coincidencias directas y habilidades complementarias. Responde solo con JSON: \texttt{\{``score'': 0--1, ``reason'': ``justificación''\}}. \\
    \addlinespace
    
    \textbf{Certificaciones} & 
    \textbf{System Message:} Eres un experto en evaluar certificaciones técnicas y profesionales. \par
    \textbf{Prompt:} Evalúa la relevancia y alineación de las certificaciones del candidato con los requisitos o habilidades técnicas del puesto. Considera certificaciones exactas, relacionadas y su vigencia. Responde solo con JSON: \texttt{\{``score'': 0--1, ``reason'': ``justificación''\}}. \\
    
    \bottomrule
  \end{tabular}
\end{table}

% =====================================================
% Tablas de resultados de evaluación del modelo (Fase 1)
% =====================================================

\section*{Resultados de evaluación del modelo - Fase 1}

Las siguientes tablas resumen los resultados cuantitativos de la primera simulación del modelo de comparación por aspectos (vacante Pragma Java Junior), contrastando los \textit{scores} producidos por el modelo y por el evaluador de RR.\,HH. para cada candidato y criterio.

% Tabla 1: Andres Felipe Chaparro Diaz y Juan Esteban Lopez
\begin{table}[H]
    \centering
    \small
    \caption{Evaluación de candidatos - Fase 1 (Parte 1)}
    \label{tab:fase1-eval-parte1}
    \begin{tabular}{|l|c|c|c|c|}
        \hline
        \textbf{Criterio} & \multicolumn{2}{c|}{\textbf{Andrés F. Chaparro}} & \multicolumn{2}{c|}{\textbf{Juan E. López}} \\
        \cline{2-5}
         & \textbf{Score Modelo} & \textbf{Score RR.\,HH.} & \textbf{Score Modelo} & \textbf{Score RR.\,HH.} \\
        \hline
        Experience        & 0\%   & 0\%   & 0\%   & 0\%   \\ \hline
        Technical Skills  & 70\%  & 70\%  & 50\%  & 40\%  \\ \hline
        Education         & 90\%  & 40\%  & 90\%  & 40\%  \\ \hline
        Responsibilities  & 0\%   & 5\%   & 10\%  & 5\%   \\ \hline
        Certifications    & -1    & -1    & -1    & 80\%  \\ \hline
        Soft Skills       & -1    & 86\%  & -1    & 67\%  \\ \hline
        Languages         & -1    & 100\% & -1    & 100\% \\ \hline
        Location          & 0\%   & 100\% & 0\%   & 100\% \\ \hline
        \textbf{Score Final} & \textbf{31\%} & \textbf{35\%} & \textbf{29\%} & \textbf{33\%} \\ \hline
    \end{tabular}
\end{table}

% Tabla 2: Javier Barrera y Gabriel Gomez
\begin{table}[H]
    \centering
    \small
    \caption{Evaluación de candidatos - Fase 1 (Parte 2)}
    \label{tab:fase1-eval-parte2}
    \begin{tabular}{|l|c|c|c|c|}
        \hline
        \textbf{Criterio} & \multicolumn{2}{c|}{\textbf{Javier Barrera}} & \multicolumn{2}{c|}{\textbf{Gabriel Gómez}} \\
        \cline{2-5}
         & \textbf{Score Modelo} & \textbf{Score RR.\,HH.} & \textbf{Score Modelo} & \textbf{Score RR.\,HH.} \\
        \hline
        Experience        & 0\%   & 0\%   & 60\%  & 80\%  \\ \hline
        Technical Skills  & 20\%  & 0\%   & 60\%  & 70\%  \\ \hline
        Education         & 90\%  & 10\%  & 100\% & 100\% \\ \hline
        Responsibilities  & 20\%  & 5\%   & 60\%  & 70\%  \\ \hline
        Certifications    & -1    & -1    & -1    & -1    \\ \hline
        Soft Skills       & -1    & 15\%  & -1    & 67\%  \\ \hline
        Languages         & -1    & 100\% & -1    & 100\% \\ \hline
        Location          & 0\%   & 100\% & 0\%   & 100\% \\ \hline
        \textbf{Score Final} & \textbf{25\%} & \textbf{12\%} & \textbf{65\%} & \textbf{80\%} \\ \hline
    \end{tabular}
\end{table}

% Tabla 3: Juanchito Bernal
\begin{table}[H]
    \centering
    \small
    \caption{Evaluación de candidatos - Fase 1 (Parte 3)}
    \label{tab:fase1-eval-parte3}
    \begin{tabular}{|l|c|c|}
        \hline
        \textbf{Criterio} & \multicolumn{2}{c|}{\textbf{Juanchito Bernal}} \\
        \cline{2-3}
         & \textbf{Score Modelo} & \textbf{Score RR.\,HH.} \\
        \hline
        Experience        & 30\%  & 30\%  \\ \hline
        Technical Skills  & 50\%  & 70\%  \\ \hline
        Education         & 80\%  & 90\%  \\ \hline
        Responsibilities  & 60\%  & 70\%  \\ \hline
        Certifications    & -1    & -1    \\ \hline
        Soft Skills       & -1    & 67\%  \\ \hline
        Languages         & -1    & 100\% \\ \hline
        Location          & 0\%   & 95\%  \\ \hline
        \textbf{Score Final} & \textbf{48\%} & \textbf{62\%} \\ \hline
    \end{tabular}
\end{table}

% Tabla resumen comparativa
\begin{table}[H]
    \centering
    \small
    \caption{Resumen comparativo de scores finales - Fase 1}
    \label{tab:fase1-eval-resumen}
    \begin{tabular}{|l|c|c|}
        \hline
        \textbf{Candidato} & \textbf{Score Modelo} & \textbf{Score RR.\,HH.} \\
        \hline
        Andrés F. Chaparro      & 31\% & 35\% \\ \hline
        Juan E. López           & 29\% & 33\% \\ \hline
        Javier Barrera          & 25\% & 12\% \\ \hline
        Gabriel Gómez           & 65\% & 80\% \\ \hline
        Juanchito Bernal        & 48\% & 62\% \\ \hline
    \end{tabular}
\end{table}
}{\chapter{Prompts utilizados} Contenido placeholder.}
\IfFileExists{appendices/C-Api.tex}{\chapter{Documentación de la API}
\label{app:api-docs}

Este apéndice presenta la documentación detallada de los endpoints de la API REST desarrollada con FastAPI. Se incluyen los métodos HTTP, las rutas, los parámetros requeridos y las descripciones funcionales de cada operación.

\section{Gestión de Hojas de Vida (HVs)}
\label{sec:api-hvs}

\begin{description}
    \item[POST /cvs]
    Procesa y guarda una nueva HV.
    \begin{itemize}
        \item \textbf{Descripción}: Recibe un archivo PDF, extrae el texto, lo estructura mediante IA y lo almacena en la base de datos.
        \item \textbf{Entrada}: Archivo.
        \item \textbf{Salida}: JSON con el ID de la HV, resumen (nombre, email, ubicación) y mensaje de confirmación.
    \end{itemize}

    \item[GET /cvs]
    Lista todas las HVs registradas.
    \begin{itemize}
        \item \textbf{Descripción}: Devuelve una lista paginada de las HVs almacenadas.
        \item \textbf{Parámetros}: skip (offset), limit (límite).
        \item \textbf{Salida}: Lista de objetos con ID, nombre, email, ubicación y fecha de creación.
    \end{itemize}

    \item[GET /cvs/\{id\}]
    Consulta el detalle de una HV.
    \begin{itemize}
        \item \textbf{Descripción}: Recupera toda la información estructurada de una HV específica.
        \item \textbf{Parámetros}: id (entero).
        \item \textbf{Salida}: Objeto completo con datos personales, educación, experiencia, habilidades, etc.
    \end{itemize}

    \item[GET /cvs/search/\{nombre\}]
    Búsqueda de HVs por nombre.
    \begin{itemize}
        \item \textbf{Descripción}: Filtra los candidatos cuyo nombre coincida parcialmente con el término de búsqueda.
        \item \textbf{Parámetros}: nombre (string).
        \item \textbf{Salida}: Lista de resultados coincidentes.
    \end{itemize}

    \item[DELETE /cvs/\{id\}]
    Elimina una HV.
    \begin{itemize}
        \item \textbf{Descripción}: Borra permanentemente una HV de la base de datos.
        \item \textbf{Parámetros}: id (entero).
        \item \textbf{Salida}: Mensaje de confirmación.
    \end{itemize}
\end{description}

\section{Gestión de Descripciones de Trabajo (Jobs)}
\label{sec:api-jobs}

\begin{description}
    \item[POST /jobs]
    Registra una nueva descripción de trabajo.
    \begin{itemize}
        \item \textbf{Descripción}: Recibe un texto con la descripción de la vacante, lo estructura con IA y lo guarda.
        \item \textbf{Entrada}: Formulario (description).
        \item \textbf{Salida}: JSON con el ID del Job, título, empresa y resumen estructurado.
    \end{itemize}

    \item[GET /jobs]
    Lista las ofertas de trabajo.
    \begin{itemize}
        \item \textbf{Descripción}: Devuelve la lista de descripciones de trabajo registradas.
        \item \textbf{Parámetros}: skip, limit.
        \item \textbf{Salida}: Lista con ID, título, empresa y ubicación.
    \end{itemize}

    \item[GET /jobs/\{id\}]
    Consulta el detalle de una oferta.
    \begin{itemize}
        \item \textbf{Descripción}: Obtiene la información completa estructurada de un Job.
        \item \textbf{Parámetros}: id (entero).
        \item \textbf{Salida}: Objeto completo con requisitos, responsabilidades, beneficios, etc.
    \end{itemize}

    \item[GET /jobs/search/\{titulo\}]
    Búsqueda de ofertas por título.
    \begin{itemize}
        \item \textbf{Descripción}: Filtra ofertas por palabras clave en el título.
        \item \textbf{Parámetros}: titulo (string).
        \item \textbf{Salida}: Lista de ofertas coincidentes.
    \end{itemize}

    \item[DELETE /jobs/\{id\}]
    Elimina una oferta de trabajo.
    \begin{itemize}
        \item \textbf{Descripción}: Borra el registro de la base de datos.
        \item \textbf{Parámetros}: id (entero).
        \item \textbf{Salida}: Mensaje de confirmación.
    \end{itemize}
\end{description}

\section{Análisis y Recomendaciones}
\label{sec:api-analysis}

\begin{description}
    \item[POST /analyze/\{cv\_id\}/\{job\_id\}]
    Ejecuta el análisis de compatibilidad.
    \begin{itemize}
        \item \textbf{Descripción}: Compara una HV y un Job específicos, calculando scores por aspecto y un puntaje global ponderado.
        \item \textbf{Entrada (Opcional)}: JSON con pesos personalizados (weights). Si se omite, usa los valores por defecto.
        \item \textbf{Salida}: Objeto con score (0-1), desglose por aspectos (score\_breakdown) y justificación detallada.
    \end{itemize}

    \item[GET /analyses]
    Historial de análisis.
    \begin{itemize}
        \item \textbf{Descripción}: Lista todos los análisis realizados en el sistema.
        \item \textbf{Salida}: Lista con ID, candidato, trabajo, score calculado y fecha.
    \end{itemize}

    \item[GET /analyses/\{id\}]
    Detalle de un análisis.
    \begin{itemize}
        \item \textbf{Descripción}: Recupera el resultado completo de un análisis previo, incluyendo las explicaciones generadas por la IA.
        \item \textbf{Salida}: JSON detallado con scores, razones y pesos utilizados.
    \end{itemize}

    \item[GET /jobs/\{id\}/top-candidatos]
    Ranking de candidatos.
    \begin{itemize}
        \item \textbf{Descripción}: Obtiene los candidatos mejor calificados para una vacante específica.
        \item \textbf{Parámetros}: id (Job ID), limit (cantidad de resultados).
        \item \textbf{Salida}: Lista ordenada descendentemente por score.
    \end{itemize}

    \item[GET /stats]
    Estadísticas del sistema.
    \begin{itemize}
        \item \textbf{Descripción}: Provee métricas generales de uso.
        \item \textbf{Salida}: Total de HVs, Jobs, Análisis y promedio global de compatibilidad.
    \end{itemize}
\end{description}

}{\chapter{Documentación de la API} Contenido placeholder.}
\makeatother

% ======= Bibliografía =======
\printbibliography

\end{document}

