% Tesis - Universidad de los Andes
% Proyecto LaTeX base
\documentclass[12pt,letterpaper,oneside]{report}

% ======= Paquetes =======
\usepackage[utf8]{inputenc}
\usepackage[T1]{fontenc}
\usepackage[spanish,es-nodecimaldot]{babel}
\usepackage{csquotes}
\usepackage{setspace}
\usepackage{geometry}
\geometry{letterpaper,margin=1in}
\usepackage{graphicx}
\usepackage{float}
\usepackage{booktabs}
\usepackage{multirow}
\usepackage{url}
\usepackage{hyperref}
\hypersetup{colorlinks=true,linkcolor=blue,citecolor=blue,urlcolor=blue}

% Bibliografía (biblatex)
\usepackage[backend=biber,style=ieee,maxbibnames=6]{biblatex}
\addbibresource{bibliography.bib}

% Espaciado
\onehalfspacing

% ======= Datos de la tesis =======
\title{Desarrollo y Evaluación de un Algoritmo de Recomendación de Hojas de Vida con NLP para Mejorar la Eficiencia en Reclutamiento}
\author{Andrés Felipe Chaparro Díaz}
\date{Universidad de los Andes\\Facultad de Ingeniería\\Programas: Ingeniería Industrial e Ingeniería de Sistemas y Computación\\2025}

% ======= Documento =======
\begin{document}

% Portada simple
\begin{titlepage}
    \centering
    {\Large Universidad de los Andes\\Facultad de Ingeniería\\Programas: Ingeniería Industrial e Ingeniería de Sistemas y Computación\\}\vspace{1.5cm}
    {\LARGE \textbf{\MakeUppercase{Desarrollo y Evaluación de un Algoritmo de Recomendación de Hojas de Vida con NLP para Mejorar la Eficiencia en Reclutamiento}}\\}\vspace{1.5cm}
    {\large Tesis de grado}\vspace{1cm}

    {\large Autor: Andrés Felipe Chaparro Díaz\\}
    {\large Asesores: Juan Fernando Pérez Bernal (Ingeniería Industrial)\\
    Rubén Francisco Manrique (Ingeniería de Sistemas y Computación)}\\[1.5cm]

    {\large Bogotá, Colombia}\vfill
    {\large 2025}
\end{titlepage}

\pagenumbering{roman}
\tableofcontents
\listoffigures
\listoftables

\chapter*{Resumen}
[Resumen de 200\,--\,300 palabras. Objetivo, método, resultados y conclusiones.]

\chapter*{Abstract}
[Abstract en inglés.]

\clearpage
\pagenumbering{arabic}

% ======= Capítulos =======
% Hacer que compile aunque no existan archivos externos (Overleaf)
\makeatletter
\IfFileExists{chapters/01-introduccion.tex}{\chapter{Introducción}
\label{chap:introduccion}

% Contexto y motivación
La gestión eficiente del talento requiere herramientas que permitan relacionar perfiles de candidatos con descripciones de trabajo de forma objetiva, transparente y reproducible. En la práctica, los equipos de reclutamiento enfrentan la revisión manual de grandes volúmenes de hojas de vida (HVs), con formatos heterogéneos (PDFs, imágenes, estilos no estandarizados) y calidad variable de contenido. Este proceso es intensivo en tiempo y recursos, puede introducir sesgos no intencionales en la preselección y dificulta priorizar candidatos óptimos con oportunidad. Este trabajo se enmarca en el nicho de tecnología: el conjunto de HVs utilizado corresponde a perfiles de ingeniería y las descripciones de cargo evaluadas están orientadas a roles de TI y desarrollo de software.

Esta tesis aborda el diseño, implementación y evaluación de un sistema de recomendación que analiza HVs y descripciones de trabajo empleando técnicas de procesamiento de lenguaje natural (NLP) y un esquema de comparación por aspectos. El sistema está compuesto por un backend que procesa y analiza la información, y un frontend que proporciona una interfaz web intuitiva para la gestión de HVs, ofertas de trabajo y visualización de resultados. El sistema toma como entradas una HV (en formato PDF) y una descripción de puesto, extrae y limpia el texto, estructura la información relevante mediante modelos de lenguaje y calcula un \emph{puntaje de ajuste} por distintos aspectos con pesos configurables para reflejar prioridades de negocio. El objetivo es apoyar una preselección inicial más eficiente, objetiva y escalable.

\section{Planteamiento del problema}
\subsection{Situación actual}
La revisión manual de HVs consume una cantidad significativa de tiempo del equipo de reclutamiento y obstaculiza la identificación rápida de candidatos con alto ajuste al puesto. La heterogeneidad de formatos (incluyendo PDFs no indexables o con bajo nivel de estructuración) y la falta de estándares dificulta la automatización y el análisis a escala. Además, la intervención humana temprana puede introducir variabilidad y sesgos no deseados.

\subsection{Situación deseada}
Se busca un prototipo funcional que:
\begin{itemize}
  \item Procese HVs en PDF, extraiga y limpie el texto de forma robusta.
  \item Estructure la información de HVs y descripciones.
  \item Calcule un \emph{puntaje de ajuste} HV--puesto y genere un ranking de candidatos.
  \item Reduzca el tiempo de revisión manual inicial  y aumente la probabilidad de identificar candidatos adecuados en etapas tempranas.
\end{itemize}

\subsection{Restricciones y enfoque}
El desarrollo se realizará con un conjunto de datos inicial de 10 hojas de vida, donde la mayoría corresponden a estudiantes o egresados de la Universidad de los Andes. De estas 10 HVs, 5 pertenecen a estudiantes activos (4 de pregrado y 1 de postgrado). Este conjunto acotado permite adoptar un enfoque iterativo de mejora continua, privilegiando una arquitectura modular, reproducible y auditable que permita refinar comparadores y pesos sin afectar la estabilidad del sistema. La limitación en el tamaño del dataset refleja el alcance inicial del prototipo y permite una evaluación controlada antes de escalar a volúmenes mayores. Adicionalmente, el estudio está acotado al dominio de tecnología; por lo tanto, la generalización de resultados hacia otros sectores (p. ej., salud, finanzas, manufactura) se discute como trabajo futuro y amenaza a la validez externa.

Como refuerzo a la evaluación técnica y para incorporar criterios de negocio propios del dominio, se contó con la participación de un profesional de recursos humanos, Juan Pablo Chaparro (25 años, alrededor de 2 años de experiencia en RR. HH.), quien colaboró en la revisión de resultados y en la interpretación de métricas del sistema. Esta retroalimentación experta contribuyó a contrastar los puntajes generados por el algoritmo con criterios prácticos de selección y a identificar casos límite relevantes para el flujo de reclutamiento.

\section{Pregunta de investigación}
¿En qué medida un sistema de evaluación de hojas de vida basado en LLMs logra alinearse con la clasificación de expertos humanos y reducir los tiempos de revisión en procesos de selección del sector tecnológico?

\section{Objetivos}
\subsection{Objetivo general}
Diseñar, implementar y evaluar un sistema de recomendación de hojas de vida basado en Grandes Modelos de Lenguaje (LLMs), con el fin de mejorar la eficiencia y efectividad del proceso de selección de talento en una empresa.

\subsection{Objetivos específicos}
\begin{enumerate}
  \item Desarrollar una metodología y herramientas para la extracción y estructuración de información a partir de HVs en formato PDF no indexable o poco estructurado.
  \item Diseñar e implementar comparadores por aspecto y un esquema de ponderación configurable para calcular el porcentaje de recomendación.
  \item Evaluar el rendimiento técnico del algoritmo y el impacto potencial del sistema en métricas operacionales de eficiencia y efectividad.
  \item Proponer lineamientos para la integración del sistema en el flujo de trabajo del equipo de reclutamiento.
\end{enumerate}

% Contribuciones
\section{Contribuciones}
\begin{itemize}
  \item Una arquitectura en capas (API, servicios, repositorios y base de datos) que soporta ingestión, análisis y persistencia de resultados, basada en \texttt{FastAPI}, \texttt{SQLAlchemy} y \texttt{SQLite}.
  \item Una interfaz web (frontend) desarrollada en \texttt{React} con \texttt{Vite} y \texttt{TailwindCSS} que permite gestionar HVs, crear ofertas de trabajo, ejecutar análisis de compatibilidad y visualizar resultados mediante gráficos interactivos.
  \item Un módulo de limpieza y extracción de texto de PDFs, y un módulo de estructuración de HVs y descripciones apoyado en orquestación de prompts utilizando \texttt{GPT-4o-mini} de Azure OpenAI.
  \item Un motor de comparación por aspectos con desglose explicativo de puntajes y esquema de pesos configurable.
  \item Un protocolo experimental y un conjunto de métricas para evaluar desempeño técnico y operacional del sistema.
  \item Un componente de validación experta con participación de un profesional de RR. HH. (Juan Pablo Chaparro), para contrastar resultados y ajustar criterios de interpretación.
\end{itemize}

\section{Consideraciones éticas y normativas}
El manejo de datos personales presentes en HVs exige medidas de seguridad, consentimiento informado y cumplimiento de la normativa de protección de datos aplicable. Asimismo, se abordará la mitigación de sesgos algorítmicos mediante transparencia de criterios, explicación de puntajes por aspecto y validaciones con casos de prueba controlados. Estas consideraciones se retoman en detalle al discutir amenazas a la validez y aspectos éticos del estudio.


}{\chapter{Introducción} Plantilla de capítulo.}
\IfFileExists{chapters/02-marco-teorico.tex}{\chapter{Marco Teórico}
\label{chap:marco-teorico}

Este capítulo presenta los fundamentos teóricos que sustentan el desarrollo del sistema de recomendación de hojas de vida. Se abordan conceptos clave para el desarrollo de la tesis principalmente en el procesamiento de lenguaje natural y en los modelos de lenguaje y extracción de información.

\section{Procesamiento de Lenguaje Natural}

El procesamiento de lenguaje natural (NLP) es una disciplina que combina lingüística, informática e inteligencia artificial para dotar a las máquinas de la capacidad de entender, interpretar y generar lenguaje humano \cite{jurafsky2023speech}. En el contexto de este trabajo, las técnicas de NLP son fundamentales para transformar documentos no estructurados (HVs en formato PDF) en representaciones estructuradas que permitan análisis automatizado.

\subsection{Extracción y Limpieza de Texto}

La extracción y limpieza de texto son procesos fundamentales en el análisis automático de documentos. La extracción se refiere a recuperar el contenido textual de un archivo, el cual puede encontrarse estructurado o representar texto a través de imágenes que requieren técnicas de reconocimiento óptico de caracteres. Una vez obtenido el contenido, la etapa de limpieza busca normalizarlo y eliminar inconsistencias propias de la digitalización o del formato, como caracteres irregulares, fragmentación de palabras o elementos considerados ruido. Estas transformaciones permiten obtener un texto coherente y uniforme que pueda ser utilizado en posteriores tareas de procesamiento y análisis.

\subsection{Pipeline de procesamiento}

 El término \emph{pipeline} se utiliza para describir la cadena de etapas que transforma entradas crudas que pueden estar en diferentes formatos (p.\,ej., imágenes en PDF o texto plano), en salidas útiles para la toma de decisiones. En sistemas de procesamiento de lenguaje natural y recomendación, un pipeline típico comienza con la extracción y limpieza de texto, continúa con la estructuración de la información (p.\,ej., conversión de texto libre a representaciones con campos definidos) y culmina en algún tipo de análisis o comparación (clasificación, ranking, cálculo de scores, etc.). Pensar el sistema como un pipeline permite razonar sobre cada etapa de forma modular---qué insumos recibe, qué transforma y qué entrega a la siguiente capa---y facilita tanto la depuración (identificar en qué punto se introducen errores) como la reproducibilidad (repetir el mismo flujo sobre nuevos datos manteniendo configuraciones constantes).


\section{Modelos de Lenguaje y Extracción de Información}

Esta sección describe cómo los modelos de lenguaje y las prácticas de diseño de \emph{prompts} permiten transformar texto libre en representaciones estructuradas útiles para el sistema: primero se presentan las capacidades de los LLMs para comprender contexto y semántica; luego se explica cómo, mediante \emph{prompt engineering} y un framework de python, se obtienen salidas en formatos controlados (p.\,ej., JSON) que alimentan los módulos de estructuración y comparaciónn.

\subsection{Modelos de Lenguaje Grandes (LLMs)}

Los modelos de lenguaje grandes son sistemas de aprendizaje profundo entrenados con grandes volúmenes de texto, capaces de aprender representaciones complejas del lenguaje y comprender su contexto y semántica. En este trabajo se emplea GPT-4o-mini, un modelo de la familia GPT (Generative Pre-trained Transformer) desarrollado por OpenAI y desplegado a través de Azure OpenAI Services. Estos modelos pueden generar respuestas estructuradas a partir de instrucciones bien definidas y adaptarse a distintos dominios o tareas con un ajuste mínimo.


\subsection{Prompt Engineering}

El \emph{prompt engineering} consiste en diseñar instrucciones que guían a los modelos de lenguaje para generar resultados precisos y estructurados. En tareas de extracción y organización de información, esta práctica permite definir con claridad el formato de salida y descomponer procesos complejos en pasos intermedios. Al incluir ejemplos dentro del propio prompt, el modelo puede adaptarse mejor al contexto y producir respuestas más consistentes. Para la orquestación de estas interacciones se emplea LangChain \cite{langchain}, un framework que permite gestionar plantillas de prompts, estructurar las respuestas mediante parsers, manejar errores y conectar distintos proveedores de modelos.




\subsection{Estrategias de Prompting}

Existen diversas técnicas para diseñar prompts efectivos dependiendo de la tarea a realizar. A continuación se describen las estrategias fundamentales utilizadas en sistemas de extracción de información:

\subsubsection{Role Prompting (Asignación de Roles)}
Esta técnica consiste en asignar una identidad o función específica al modelo al inicio de la interacción (p.\,ej., "Eres un experto en recursos humanos"). Al definir un rol, se condiciona el tono, el vocabulario y el enfoque de la respuesta, alineando la generación del modelo con las expectativas del dominio específico de la tarea \cite{jurafsky2023speech}.

\subsubsection{Zero-Shot Prompting}
En el \emph{zero-shot prompting}, se presenta al modelo una tarea sin proveer ejemplos previos de la solución esperada. El modelo debe inferir cómo resolver el problema basándose únicamente en su entrenamiento previo y en la descripción textual de la instrucción. Esta estrategia es efectiva con modelos de gran capacidad (LLMs) que han generalizado conocimiento sobre múltiples tareas durante su fase de entrenamiento.

\subsubsection{Structured Output Prompting (Salida Estructurada)}
Esta estrategia se enfoca en restringir el formato de la respuesta generada. En lugar de solicitar texto libre, se instruye al modelo para que genere su salida siguiendo un esquema rígido, comúnmente JSON o XML. Esto es crítico para integrar modelos de lenguaje en sistemas de software automatizados, ya que permite que la salida del modelo sea parseada y validada programáticamente sin necesidad de post-procesamiento complejo de lenguaje natural.

\subsection{LangChain}

LangChain es un framework diseñado para facilitar la construcción de aplicaciones basadas en modelos de lenguaje mediante la definición de cadenas de procesamiento, plantillas de prompts reutilizables y mecanismos de estandarización de entradas y salidas. Además, incorpora herramientas para validar formatos de salida, integrar proveedores externos de modelos y gestionar parámetros de ejecución de forma consistente. Estas características lo convierten en una plataforma adecuada para desarrollar sistemas que requieran orquestar interacciones complejas con modelos de lenguaje de manera reproducible y controlada.\cite{langchain}

}{\chapter{Marco Teórico} Contenido placeholder.}
\IfFileExists{chapters/03-estado-del-arte.tex}{\chapter{Estado del Arte}
\label{chap:estado-del-arte}

\section{Sistemas de Recomendación de Talento}
Revisión de enfoques académicos e industriales para emparejamiento HV--puesto.

\section{Extracción y Estructuración de HVs}
Técnicas, esquemas de información y retos en documentos no estructurados.

\section{Evaluación y Métricas}
Prácticas de evaluación, datasets y métricas (precisión, ranking, correlación, etc.).


}{\chapter{Estado del Arte} Contenido placeholder.}
\IfFileExists{chapters/04-metodologia.tex}{\chapter{Metodología}
\label{chap:metodologia}

\section{Preguntas de Investigación}

Este trabajo busca responder a la pregunta principal planteada en la Introducción: ¿Cómo el diseño e implementación de un sistema de recomendación de hojas de vida basado en procesamiento de lenguaje natural mejora la efectividad del proceso de selección de talento?

Derivamos las siguientes preguntas específicas:
\begin{enumerate}
  \item ¿Con qué calidad se estructuran las hojas de vida y las descripciones de trabajo (por rubro/campo) al transformarlas a JSON?
  \item ¿En qué medida el \emph{score} de compatibilidad del sistema y sus explicaciones por aspecto son considerados adecuados por un profesional de RR.\,HH.?
  \item ¿En qué medida coinciden los rankings de candidatos producidos por el sistema con el ranking emitido por un profesional de RR.\,HH.?
\end{enumerate}

Hipótesis asociadas:
\begin{itemize}
  \item \textbf{H1 (Estructuración CV/Job)}: La estructuración automática alcanza puntajes altos (0--100) por rubro al compararse con las fuentes originales (PDF del CV y texto del Job).
  \item \textbf{H2 (Adecuación de score y explicación)}: Un profesional de RR.\,HH. califica con valores altos (0--100) tanto el \emph{score} del sistema como las razones por aspecto (experiencia, responsabilidades, habilidades, etc.).
  \item \textbf{H3 (Consistencia de ranking)}: El orden de candidatos del sistema muestra coincidencias relevantes con el ranking emitido por el profesional de RR.\,HH.
\end{itemize}

\section{Diseño Experimental}

\subsection{Contexto y dataset}
Se emplea un conjunto acotado de 10 hojas de vida (la mayoría de estudiantes/egresados de la Universidad de los Andes; 5 estudiantes activos) y descripciones de cargos en tecnología (desarrollo de software y TI). Este tamaño permite un ciclo iterativo de prueba y mejora controlado.

\subsection{Variables}
\textbf{Independientes}:
\begin{itemize}
  \item Descripción de cargo (tarea) frente a la cual se evalúan candidatos.
  \item Conjunto de hojas de vida analizadas por tarea.
\end{itemize}
\textbf{Dependientes}:
\begin{itemize}
  \item Calidad de estructuración (0--100) por rubro/campo en CVs y Jobs.
  \item Calificación (0--100) del \emph{score} del sistema por análisis.
  \item Calificación (0--100) de las razones/explicaciones por aspecto.
  \item Comparación de rankings entre experto y sistema (coincidencias/diferencias).
\end{itemize}

\subsection{Métricas}
\begin{itemize}
  \item \textbf{Estructuración de CVs y Jobs}: Puntuación 0--100 por rubro al contrastar JSON contra el PDF (CV) o texto (Job). Se reportan promedios por rubro y un promedio global por documento.
  \item \textbf{Adecuación del modelo (por análisis)}:
    \begin{itemize}
      \item Calificación 0--100 del \emph{score} de compatibilidad producido por el sistema.
      \item Calificación 0--100 de la explicación por aspecto (experiencia, responsabilidades, habilidades técnicas, habilidades blandas, certificaciones, idiomas, ubicación).
    \end{itemize}
  \item \textbf{Ranking experto vs.\ sistema}: Se comparan las listas ordenadas de candidatos por análisis y se registran coincidencias y diferencias principales (sin aplicar métricas de precisión/recuperación).
\end{itemize}

\subsection{Procedimiento}
\begin{enumerate}
  \item \textbf{Preparación}: Limpieza y estructuración de HVs y descripciones (pipeline de extracción; ver capítulos previos). Registro de versiones de prompts y configuraciones.
  \item \textbf{Análisis automático}: Para cada combinación \emph{(CV, Job)} se ejecuta el análisis con pesos \emph{default}. Se almacenan \emph{score}, desglose por aspecto y tiempos.
  \item \textbf{Estructuración (validación manual)}: Para CVs y Jobs se revisa rubro por rubro el JSON estructurado contra la fuente (PDF o texto) y se asigna una calificación 0--100 por rubro. Se agregan promedios por rubro y globales por documento.
  \item \textbf{Evaluación experta del modelo}: Un profesional de RR.\,HH., Juan Pablo Chaparro (25 años, \(\sim\)2 años de experiencia), realiza 4 análisis; en cada uno evalúa 5 hojas de vida (20 evaluaciones en total). Para cada análisis:
    \begin{itemize}
      \item Califica 0--100 el \emph{score} de compatibilidad entregado por el sistema.
      \item Califica 0--100 las razones por aspecto (experiencia, responsabilidades, habilidades, certificaciones, idiomas, ubicación).
      \item Emite un ranking de candidatos y se compara cualitativamente con el ranking del sistema (coincidencias/diferencias).
    \end{itemize}
\end{enumerate}

\subsection{Amenazas y controles}
\begin{itemize}
  \item \textbf{Tamaño muestral reducido}: Se reportan resultados con intervalos y se evita sobreinterpretación; se propone ampliar dataset como trabajo futuro.
  \item \textbf{Variabilidad del modelo de lenguaje}: Se fija configuración conservadora (temperatura baja) y se almacenan salidas intermedias para trazabilidad.
  \item \textbf{Sesgo de dominio}: El estudio se acota a roles de tecnología; se declara la validez externa como amenaza y se discute su impacto.
\end{itemize}

\section{Reproducibilidad}

\subsection{Entorno y dependencias}
Se utiliza Python 3.11, FastAPI, SQLAlchem}, PyMuPDF, NLTK, LangChain y cliente de Azure OpenAI. Las versiones se especifican en \texttt{requirements.txt}. 

\subsection{Modelos y configuración}
Se emplea \texttt{GPT-4o-mini} vía Azure OpenAI con variables de entorno (\texttt{.env}). Se recomienda fijar temperatura baja y registrar \emph{prompts} en los apéndices. Las respuestas estructuradas (CVs/Jobs) y resultados de análisis se persisten en \texttt{cv\_system.db}.

\subsection{Datos y trazabilidad}
Los PDFs de HVs y textos de descripciones se versionan como artefactos de datos (o referencias anonimizadas). Se guarda:
\begin{itemize}
  \item Texto extraído y versiones de limpieza.
  \item JSON estructurado por entidad (CV/Job).
  \item Resultados de análisis: \emph{score}, desglose por aspecto, tiempos.
  \item Anotaciones/orden del experto de RR.\,HH.
\end{itemize}




}{\chapter{Metodología} Contenido placeholder.}
\IfFileExists{chapters/05-diseno-arquitectura.tex}{\chapter{Diseño y Arquitectura}
\label{chap:diseno}

\section{Visión General}
Arquitectura en capas: API (endpoints), servicios, repositorios y base de datos.

\section{Módulos Principales}
\begin{itemize}
  \item Limpieza y extracción de texto (PDF \textrightarrow{} texto).
  \item Estructuración de HVs y descripciones (texto \textrightarrow{} JSON).
  \item Comparadores por aspecto y cálculo de score final con pesos.
\end{itemize}

\section{Decisiones de Diseño}
Racionales, \textit{trade-offs} y consideraciones de escalabilidad y mantenibilidad.


}{\chapter{Diseño y Arquitectura} Contenido placeholder.}
\IfFileExists{chapters/06-implementacion.tex}{\chapter{Implementación}
\label{chap:implementacion}

\section{Tecnologías y Entorno}
Lenguajes, librerías y configuración (FastAPI, SQLAlchemy, PyMuPDF, etc.).

\section{Módulos y Detalles}
Detalles de implementación por módulo y contratos de interfaces.

\section{Persistencia y API}
Modelos de datos, endpoints y validación.


}{\chapter{Implementación} Contenido placeholder.}
\IfFileExists{chapters/07-experimentos-resultados.tex}{\chapter{Experimentos y Resultados}
\label{chap:experimentos}

\section{Diseño Experimental}
Datasets, particiones, métricas y configuración de pruebas.

\section{Resultados}
Tablas y figuras con resultados principales y análisis.

\section{Ablaciones y Sensibilidad}
Impacto de pesos por aspecto y variantes de comparadores.


}{\chapter{Experimentos y Resultados} Contenido placeholder.}
\IfFileExists{chapters/08-discusion-amenazas.tex}{\input{chapters/08-discusion-amenazas}}{\chapter{Discusión y Amenazas a la Validez} Contenido placeholder.}
\IfFileExists{chapters/09-conclusiones-trabajo-futuro.tex}{\chapter{Conclusiones y Trabajo Futuro}
\label{chap:conclusiones}

\section{Conclusiones}
Resumen de hallazgos y respuesta a preguntas de investigación.

\section{Trabajo Futuro}
Extensiones propuestas: nuevos comparadores, datasets ampliados, despliegue y uso en producción.


}{\chapter{Conclusiones y Trabajo Futuro} Contenido placeholder.}
\makeatother

% ======= Apéndices =======
\appendix
\makeatletter
\IfFileExists{appendices/A-datasets-y-prompts.tex}{\chapter{Datasets y Prompts}

\section{Descripción de Datasets}
Estructura, licencia y criterios de selección.

\section{Prompts y Configuración}
Listado de prompts y parámetros usados para estructuración y comparación.


}{\chapter{Datasets y Prompts} Contenido placeholder.}
\IfFileExists{appendices/B-instrumentos-y-scripts.tex}{\chapter{Instrumentos y Scripts}

\section{Instrumentos de Evaluación}
Formularios, rúbricas y criterios de anotación.

\section{Scripts}
Listado de scripts para preprocesamiento, ejecución de experimentos y análisis.


}{\chapter{Instrumentos y Scripts} Contenido placeholder.}
\makeatother

% ======= Bibliografía =======
\printbibliography

\end{document}

